\chapter{Transport domain}

\section{Automated planning}

Planning is usually defined as the reasoning side of acting -- an abstract deliberation
process that chooses and organizes actions by anticipating their outcomes. \cite[Section~1.1]{Ghallab2004}
It seems only natural that we want to have computers to do this strenuous activity for us.
Automated planning is an attempt at just that -- it is an area of Artificial Intelligence (AI) that
studies the planning process computationally. \cite[Section~1.1]{Ghallab2004}

\TODO{Define domain and problem, planner?}

Unfortunately, the specific situations in which we want to use automated planning are very diverse --
from devising a sequence of actions to shut down a nuclear power plant to planning the robotic arm
movements in an assembly line or devising the complex motor activations for space aircraft positioning.
Due to this, people are often interested in domain-independent planning, where the planner gets information
about both the domain and the specific problem and attempts to devise a plan using only the provided knowledge
and its previously built-in processes. \cite[Section~1.3]{Ghallab2004}

On the other hand, domain-specific planning, where domain knowledge has been built into the planner,
has obvious advantages when solving problems in that domain -- all the while being useless on problems of other
domains. \cite[Section~1.3]{Ghallab2004}

\subsection{Planning model}

As a basis for the later-defined representation of planning, we first define
a conceptual model similar to the restricted model in \cite[Section~1.4, Section~1.5]{Ghallab2004}.

\begin{defn}[State-transition system]\label{defn:state-transition-sys}
A (restricted) state-transition system is a 3-tuple $\Sigma = (S, A, \gamma)$, where:
\begin{itemize}
\item $S = \{s_1, s_2, \ldots\}$ is a finite and fully observable set of states,
\item $A = \{\noop, a_1, a_2, \ldots\}$ is a finite set of actions;
\item $\gamma: S \times A \to S \cup \{\emptyset\}$ is a state-transition function,
where $\forall s \in S : \gamma(s, \noop) = s$,
and $\forall s \in S\,\exists a \in A : \gamma(s, a) \neq \emptyset$; and
\item $\Sigma$ is static and offline,
it only changes when an action is applied to it and does not change while planning.
\end{itemize}
In the basic version, all actions have no duration.
\end{defn}

\noindent \TODO{define goals}

\begin{defn}[Planning problem]\label{defn:planning-problem}\cite[Part~I]{Ghallab2004}
A planning problem is a 5-tuple $\mathbb{P} = (S, A, \gamma, s_0, g)$, where:
\begin{itemize}
\item $(S, A, \gamma)$ is a state-transition system;
\item $s_0 \in S$ is an initial state; and
\item $g \subseteq S$ is a set of goal states.
\end{itemize}
\end{defn}

\noindent For notation purposes, we define $[k] := \{1, 2, \ldots, k\}$ for all $k \in \N$

\begin{defn}[Plan]\label{defn:plan}\cite[Section~1.5]{Ghallab2004}
For a planning problem $\mathbb{P} = (S, A, \gamma, s_0, g)$,
a plan is a finite ($k \in \N$) sequence of actions $(a_1, a_2, \ldots, a_k)$ where
$\forall i \in [k] : a_i \in A$ such that
$\forall i \in [k] : \gamma(s_{i-1}, a_i) = s_i$ and $s_k \in g$.
\end{defn}

\noindent A basic planning model consists of three components:

\begin{itemize}
\item A \textit{state-transition system} $\Sigma$ that evolves by its state-transition function using the actions
it receives.
\item A \textit{controller} -- given an input state $s \in S$ provides an action $a \in A$ as output according
a plan.
\item A \textit{planner} -- uses a description of $\Sigma$ to synthesize a plan for the controller
to execute in order to achieve the objective.
\end{itemize}

\TODO{image of the system}

\subsection{Classical planning}

There are several theoretical representations of planning problems.

\subsection{Neoclassical planning}

\subsection{Temporal planning}

\section{Transport domain definition}

\TODO{Focus on a few domain variants, just show the others}

\subsection{Key insights}

\TODO{degenerate cases, etc.}

\section{The Vehicle Routing Problem}

\TODO{mention Dantzig}

\subsection{Problem formulation}

\subsection{Comparison to the Transport domain formulation}

\subsection{Formulating the sequential Transport domain as a VRP problem}

\TODO{Ghallab 8.3}
