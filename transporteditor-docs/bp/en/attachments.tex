\chapwithtoc{Attachments}

\section*{CD contents}\label{cd-contents}

The compact disk attached to this thesis contains the following:

\begin{itemize}
\item \verb+thesis.pdf+ -- the PDF version of this thesis;
\item \verb+TransportEditor/+ -- the \TODO{version} release of the TransportEditor software, also obtainable at:\\
\url{https://github.com/oskopek/TransportEditor/releases}. Contains:
\begin{itemize}
\item \verb+datasets/+: the datasets used at IPC;
\item \verb+docs/+: JavaDoc documentation and specification for TransportEditor;
\item \verb+sources/+: sources of TransportEditor;
\item \verb+tools/+: the benchmarker and other tools used; and
\item \verb+TransportEditor-jar-with-dependencies.jar+: the executable jar with all dependencies bundled in.
\end{itemize}
\item \TODO{other files on CD}
\end{itemize}

\subsubsection*{TransportEditor Developer JavaDoc}\label{transporteditor-developer-javadoc}

The generated API documentation using JavaDoc is also available on the attached CD, in the folder
\verb+TransportEditor/docs/javadoc+.

\newpage

\section*{TransportEditor User Manual}\label{transporteditor-user-manual}

%%%% manual_en
Welcome to TransportEditor's manual!

TransportEditor is a planning dashboard for the Transport planning domain.
You can use it to study various Transport domain variants,
create, edit and visualize problem instances, after which you can
run various planners and validators and visualize the resulting plans.

How TransportEditor works and what it can do:

\begin{enumerate}
\item Create a new session, create or load a domain.

\item Create/load a problem. TransportEditor will display the road graph for you.
You can use the buttons on the right to edit the graph and its elements.

\item Go to ``Session'' $\to$ ``Set Planner'' to set a planner to use. Likewise, set a validator in
``Session'' $\to$ ``Set Validator''

\item{Click on the ``Plan'' button to run the planner. After some time, the planning will end and you can
see the generated plan in the table on the right. There are several plan views available, f.e. the Gantt chart.}
\end{enumerate}

Now you can edit the graph, the vehicles, packages, rerun your planner and test the outcome. Happy planning!

Read the following sections if you want to learn to work with TransportEditor quickly and effectively.

\subsection*{Getting help}
If you can't find the answer to your question or request in this document
(located in the menu, in ``Help'' $\to$ ``Help''),
feel free to submit a pull request/issue in
the \href{https://gitlab.com/oskopek/TransportEditor}{GitLab repository} of TransportEditor,
post a question on \href{https://stackoverflow.com}{StackOverflow}, or email me personally.
You can find all the needed information in the menu, in ``Help'' $\to$ ``About''.

\subsubsection*{Changing the language}
If you want to run a translated version of TransportEditor,
set your system locale accordingly and restart TransportEditor.

If you want to try out a translated version (on Linux) without changing your system locale permanently,
run:\\
\verb+LC_ALL="LOCALE" java -jar TransportEditor.jar+\\
Substitute \texttt{LOCALE} for any locale available on your system. To view all available locales,
run \verb+locale -a+.

TransportEditor currently does not support changing the language on the fly.

\subsection*{Loading and saving}
\subsubsection*{Session}
The session is an abstraction over all the user workspace. It encompasses all your currently loaded data and options,
including the domain, problem, plan, planner, and validator.

You can create a new session by clicking on ``Session'' $\to$ ``New''.

Session can be saved to an XML file using ``Session'' $\to$ ``Save'' or ``Session'' $\to$ ``Save As''.
After that, you can come back to your session any time by using ``Session'' $\to$ ``Load'', without
having to load all the individual parts again.

\subsubsection*{Domain}
Loading a domain from its PDDL file is as simple as clicking on ``Domain'' $\to$ ``Load''.
A file chooser dialog will pop up, where you can select the correct file.

\subsubsection*{Domain variant creator}
After clicking on ``Domain'' $\to$ ``New'', the domain variant creator will help
you adjust the constraints and features available in your domain.
The base of the domain includes a few predicates and functions, along with:

\begin{itemize}
\item the graph

\item distances of edges

\item locatables (packages, vehicles, …)
\end{itemize}

And actions:

\begin{itemize}
\item Pick up (a package into a vehicle)

\item Drop (a package from a vehicle)

\item Drive (a vehicle between two nodes on an edge)
\end{itemize}

Each action has an associated cost, and the minimization function is by default the total cost of a plan.

There two basic domain types you can select:

\begin{itemize}
\item Sequential -- All actions are disjunct in time, the total-duration (sum of action durations) is minimized by default.

\item Temporal -- Introduces start/end times of actions,
preconditions/effects are now checked with temporal reasoning (at start/at end/over all),
new minimization function -- total-time.
\end{itemize}

There are several optional constraints you can place on your model:

\begin{itemize}
\item the absence of capacity (vehicles have no maximum capacity)

\item fuel (introduces petrol stations, max fuel capacity of a vehicle and it's current status.
Also introduces an additional edge-weight, fuel-demand -- equally long edges can have different
fuel-demands, for example)

\item numerical minimization -- currently not implemented. Sorry for the inconvenience.
\end{itemize}

You can select any subset of these constraint packages, but only a single cost function to go with them.
The domain variant creator will then create your chosen domain and TransportEditor enables you to export it to PDDL,
so that you can use it in different (mainly domain-independent) planners.

\subsubsection*{Problem}
Loading a problem from its PDDL file is as simple as clicking on ``Problem'' $\to$ ``Load''.
A file chooser dialog will pop up, where you can select the correct file.

Saving and creating new problems works quite simply with the appropriate button in the ``Problem'' menu.

\subsubsection*{Plan}
Loading a plan from its PDDL file is as simple as clicking on ``Plan'' $\to$ ``Load''.
A file chooser dialog will pop up, where you can select the correct file.

Saving and creating new plans works quite simply with the appropriate button in the ``Plan'' menu.
Plans can also be changed by the planner (see the ``Plan'' button on the right) and by the user, in the plan view
table at the bottom right -- either by rearranging the plan actions or changing their start time.

\subsubsection*{Demo data}
You can try all of this out on the datasets gathered from the IPC competitions, located in the folder \texttt{datasets/}.

\subsection*{Editing the problem}
Left-clicking and dragging will move locations on the screen. To attempt a new redraw of the graph,
press the ``Redraw'' button. If the layout doesn't get redrawn properly, try moving a few nodes from their
positions to destabilize the algorithm and press redraw again.

You can select graph elements by left-clicking and dragging in the graph area, creating a selection region.
The buttons on the right will get enabled/disabled based on the selected elements in the graph.
To be aware of all the selection options, it is recommended to read through the
Shortcut quick-tip manual, located in ``About'' $\to$ ``Shortcut quick tips''. To deselect nodes,
press Escape (Esc) or click on a point in the graph containing no element.

The ``Lock'' button disables all graph editing. Clicking it again will unlock the graph.

\subsection*{Planning}
To set a planner, go to the ``Session'' $\to$ ``Set Planner'' menu.

\subsubsection*{External planners}
In the ``Set Planner'' menu, you can choose to set an external planner. An external planner is either an
executable's name that is present in the filesystem executable path (based on the environment variable PATH)
or directly a path to an executable file.

The planner executable must follow a few rules (usually these are enforced by wrapping your planner in a shell script
that enforces these rules):

\begin{itemize}

\item Only the resulting plan is written to standard output.

\item All logging, debugging and error messages are written to standard error output.

\item The executable must exit with a 0 return code if and only if planning succeeded and a plan was output to stdout.

\item The executable must take two file path parameters -- the domain in PDDL and the problem in PDDL.

\begin{itemize}

\item You can specify a parameter template while setting the external planner. For example, you can specify \texttt{fast-downward}
to be the executable and its parameters to be \verb+{0} {1} --search "astar(lmcut())"+. The template \verb+{0}+ will get
substituted for the domain filename at runtime, the \verb+{1}+ will be the problem filename.

\end{itemize}

\end{itemize}

\subsubsection*{Internal planners}
Unfortunately, TransportEditor does not currently contain any internal planners.

\subsubsection*{Validating your plan}
To set a validator, go to the ``Session'' $\to$ ``Set Validator'' menu.

The ``Sequential Validator'' is a simple validator used for sequential domains that checks basic
properties of the problem state, like predicates holding before and effects holding after applying an action.
For more detailed information, see the class \texttt{SequentialValidator}.

\subsubsection*{External validators}
In the ``Set Validator'' menu, you can choose to set an external validator. An validator planner is either an
executable's name that is present in the filesystem executable path (based on the environment variable PATH)
or directly a path to an executable file.

The validator executable must follow a few rules:

\begin{itemize}
\item All logging, debugging and error messages are written to standard error output or standard output.

\item The executable must exit with a 0 return code if and only if the plan is valid.

\item The executable must take three file path parameters -- the domain in PDDL, the problem in PDDL and the plan in a
VAL-like format.

\begin{itemize}

\item You can specify a parameter template while setting the external validator.
For example, you can specify \texttt{val} to be the executable and its parameters to be ``\verb+{0} {1} {2}+''.
The template \verb+{0}+ will get substituted for the domain filename at runtime, the \verb+{1}+ will be the problem filename and
\verb+{2}+ will be the plan filename.

\end{itemize}

\end{itemize}

\subsection*{Plan views}
On the lower right-hand side when a plan is loaded, you can see a table with the plan's actions. If the loaded domain
is sequential, that table displays basic action info and its first column is draggable - you can reorder the plan's
actions by dragging them from one place to another in the table. If the domain is temporal, you can double click and
edit the action start time in the table.

Both of these tables are filterable -- if you right click on the table's headers a small dialog popup will show up
and you can choose which values to show and which to hide. These choices will get propagated to the other plan views
and they too will get filtered (if possible).

One of the alternative table views is the Gantt chart. It displays action objects on the Y-axis and time on the X-axis.
The contents of the chart are individual actions involving the given action objects at the given time. The color
corresponds to a type of action.

\subsection*{Stepping through a plan}
When a plan is loaded, there is an option of stepping through the plan's actions and visualizing the intermediate
problem state. To enable this, press the ``Show steps'' button in the panel on the right -- this will disable graph editing.
Do note that stepping through the plan does not change the
actual problem -- if you press ``Hide steps'', the original problem will appear.

When stepping through a plan, a new panel will display, showing navigation buttons.
You can move in the plan in three different ways:

\begin{itemize}

\item Using the up (``$\wedge$'') and down (``$\vee$'') arrow buttons. This will de-apply or apply the previous
or next action in the plan. Do note, that in temporal plans, this might not necessarily be the action
above or beneath the currently selected one, although it probably will be.

\item Selecting the appropriate action in the plan table. The plan state will change to before, during or after the
selected action is applied -- depending on which time button is selected (``Start'', ``Middle'' or ``End'').

\item Selecting a specific time in the text field on the right. You can input any positive number and the plan will move to
that time step. It will either apply the ``at start'' effects of actions starting at the selected time or not, depending
on the selection of the ``Apply starts'' button. If it is selected, the ``at start'' effects will be applied.
Otherwise, only ``at end'' effects of actions ending at the selected time will be applied to the plan state.

\end{itemize}
%%%% END manual_en

\newpage

\section*{TransportEditor Developer Documentation}\label{transporteditor-developer-documentation}

%%%%%%%%% README
TransportEditor aims to be a problem editor and plan visualizer for the Transport domain from the IPC 2008.
The goal is to create an intuitive GUI desktop application for making quick changes and re-planning,
but also designing a new problem dataset from scratch.

\subsection*{Getting help}

\textbf{For users looking for help}: a manual describing all possible features of TransportEditor is available in the app itself:
in the upper menu bar item \textbf{Help $\to$ Help}.

Post any development questions or comments you may have on Stack Overflow and/or don't hesitate to
\href{https://gitlab.com/oskopek/TransportEditor/issues}{open an issue}.

\subsection*{Running TransportEditor releases}
\begin{itemize}
\item Download a release from:\\ \url{https://github.com/oskopek/TransportEditor/releases}

\item Note: TransportEditor uses \href{http://semver.org/}{semantic versioning}.

\item To directly run TransportEditor, download the \textbf{executable JAR file}: \texttt{TransportEditor-VERSION-jar-with-dependencies.jar}

\item If you want to run a release, just try:\\ \texttt{java -jar TransportEditor-VERSION-jar-with-dependencies.jar}
\end{itemize}

\subsection*{Building \& running TransportEditor}\label{readme-building}

\begin{itemize}
\item See the section (further down) on How-to setup your \textbf{build environment} first.

\item \textbf{Recommended}: \texttt{mvn clean install -DskipTests}

\item To run \textbf{unit tests}: \texttt{mvn clean install}

\item To run \textbf{integration tests}: \texttt{mvn clean install -Pit}

\item To \textbf{clean}, run: \texttt{mvn clean}

\item To build \textbf{docs} too: \texttt{mvn clean install -Pdocs}

\item \textbf{Run TransportEditor}:
\begin{itemize}
\item If you followed the build environment setup and you want to run your version of TransportEditor,
run the command \texttt{mvn exec:java} from the \texttt{transporteditor-editor} module directory.

\item If you want to run a translated version of TransportEditor, set your system locale accordingly and restart TransportEditor. If you want to try out a translated version (on Linux), try:\\ \verb+LC_ALL="LOCALE" java -jar TransportEditor.jar+\\
Substitute \texttt{LOCALE} for any locale available on your system. To view all available locales,
run \verb+locale -a+.
\end{itemize}
\end{itemize}

\subsection*{Setup your build environment}
\begin{enumerate}

\item \textbf{Install Git}

\begin{itemize}

\item Fedora: \texttt{sudo dnf install git}

\item Ubuntu: \texttt{sudo apt-get install git}

\item Windows: Download and install Git for Windows at:\\ \url{https://git-scm.com/downloads}

\end{itemize}

\item \textbf{Install git-lfs} from \href{https://git-lfs.github.com/}{https://git-lfs.github.com/}

\item \textbf{Install Java8 JDK} -- \href{http://www.oracle.com/technetwork/java/javase/downloads/index.html}{Oracle JDK Downloads} -- Select: Java Platform (JDK)

\begin{itemize}

\item \textbf{NOTE}: You need \texttt{jdk-8u40} or newer (JavaFX 8 dependency).

\end{itemize}

\item \textbf{Install Maven} -- preferably the latest version you can.
Usually, your distribution's package management repository is enough:

\begin{itemize}

\item Fedora: \texttt{sudo dnf install mvn}

\item Ubuntu: \texttt{sudo apt-get install maven}

\item Windows: See: \href{http://maven.apache.org/guides/getting-started/windows-prerequisites.html}{Maven on Windows}
 and \href{http://maven.apache.org/download.cgi}{Maven Downloads}.

\end{itemize}

\item \textbf{Fork the repository} -- Create a fork of the \href{https://gitlab.com/oskopek/TransportEditor/}{oskopek/TransportEditor repository}
(under the logo) on GitLab, usually the fork will be called: \texttt{yourusername/TransportEditor}.

\item \textbf{Clone the your fork} -- run:\\ \texttt{git clone \href{https://gitlab.com/yourusername/TransportEditor.git}{https://gitlab.com/yourusername/TransportEditor.git}}\\
Or, preferably, use SSH:\\ \texttt{git clone \href{mailto:git@gitlab.com}{git@gitlab.com}:yourusername/TransportEditor.git}

\item \textbf{Pull the LFS files} -- run: \texttt{git lfs pull}

\item \textbf{Run the build} (see: \nameref{readme-building})
\end{enumerate}

\subsection*{Short design description}
The model for the Transport domain is pretty complicated
because it handles:

\begin{itemize}
\item Multiple variants of the Transport domain

\item Planning and visualization with the same model
\end{itemize}

That's what this short section is for -- describing the ideas behind the model, so that reading the code
afterward is easier. The model is split into 4 parts:

\begin{itemize}
\item Session
\item Domain
\item Problem
\item Plan
\end{itemize}

\subsection*{Plan}
The plan consists of an ordered list of actions.
There are two types of plans:

\begin{itemize}
\item Sequential -- these plans are strictly linear, actions do not overlap (imagine simple linked list).
\item Temporal -- every action in this plan has a time interval in which it takes place.
This plan is basically a set of intervals with associated actions. For storing it, we use an
\href{https://en.wikipedia.org/wiki/Interval_tree}{Interval tree},
which allows efficient access to actions given a time or time range.
\end{itemize}

\subsubsection*{Visualizing plans}

There are currently two ways to visualize both plan types:

\begin{itemize}
\item Simple list -- both sequential and temporal versions look similar. Both are filterable by right clicking on the headers.
\begin{itemize}
\item Sequential: uses a simple drag-and-drop reorderable table of action arguments.
See the screenshot on the top of the README for a preview. Is redrawn completely after every change.

\item Temporal: in contrast to the sequential variant, this one cannot be reordered by dragging. The start times can, however,
be edited, which will result in the table reordering itself. Is not redrawn completely, adjusts its internal state and
redraws the necessary parts.
\end{itemize}

\item Gantt chart -- both sequential and temporal versions look alike, resembling an XY chart, the X-axis being the time
axis and the Y-axis having all action objects. Both are redrawn every time the plan changes or its filter
in the simple list is changed.
\begin{itemize}
\item Sequential: using it to visualize sequential plans is not very interesting, as it offers almost no added insights on top of the simple list

\item Temporal: when visualizing temporal plans with a Gantt, we can observe the synchronicity of planned actions
and, to some extent, the cooperation of individual actors
\end{itemize}
\end{itemize}

\subsubsection*{Persisting plans}
Using string manipulation and built-in constants and format, it is persisted into a VAL-like format.
For parsing, we assume a correct and valid VAL-like plan. A very simple string manipulation and Regex-based approach
is used for both temporal and sequential plans. Additionally, a simple \href{http://www.antlr.org/}{ANTLR} grammar
is used in some places. See the \texttt{persistence} package for details.

\subsection*{Problem}
The problem is basically a graph (with multiple possible ``layers'', f.e. fuel) and a vehicle and package map.

Currently, we use \href{http://graphstream-project.org/}{GraphStream} for both the data storage and visualization of the graph.
Apart from nodes and edge arrows, everything else is visualized as
``\href{http://graphstream-project.org/doc/Tutorials/Graph-Visualisation/#sprites}{sprites}''.

Fuel is added as different graph edge type (\verb+FuelRoad+ instead of \verb+DefaultRoad+) and a domain label change
(see \verb+PddlLabel+s in the domain).
If the domain is fuel enabled, the fuel properties of locations, roads, and vehicles else will be displayed.

\subsubsection*{Visualizing problems}

Problem visualization does not fundamentally differ between different domains and problems.
Some problem tooltips/properties might dis/appear when changing the domain type.

The graph is automatically laid out using a \texttt{SpringBox} algorithm from GraphStream
for a given time and then switched to a manual layout.

\subsubsection*{Persisting problems}
Both rule pages of \href{http://icaps-conference.org/ipc2008/deterministic/CompetitionRules.html}{IPC-6}
and \href{https://helios.hud.ac.uk/scommv/IPC-14/rules.html}{IPC-8}
specify PDDL 3.1 as their official modeling language (language for domain
and problem specification).
Daniel L. Kovacs proposed an updated and corrected BNF (Backus-Naur Form)
\href{https://helios.hud.ac.uk/scommv/IPC-14/repository/kovacs-pddl-3.1-2011.pdf}{definition of PDDL 3.1}.

Using a \href{http://freemarker.org/}{Freemarker} template and a lot of string manipulation it is persisted into PDDL.
For parsing, we assume a correct and valid problem and use a formal grammar written in \href{http://www.antlr.org/}{ANTLR}
to parse PDDL into a generated code structure provided by ANTLR and the \texttt{maven-antlr-plugin}. The same grammar as for
domains is used. See the \texttt{persistence} package and the \texttt{src/main/antlr4} folder for details.

\subsection*{Domain}
There is basically only one domain type: \texttt{VariableDomain} (we also have the notion of a \texttt{SequentialDomain},
but it is basically just an in-code hardcoded equivalent of loading the sequential Transport domain PDDL
into a \texttt{VariableDomain}).

The domain contains flags (labels), telling us which ``layers'' are enabled and which are not.
The UI, validator, IO, and planner all take these into account.
It also contains methods for action creation using their correct domain-specified definitions
and provides other useful data (predicates, functions, \ldots).

\subsubsection*{Visualizing domains}

Domains are not visualizable per se.

\subsubsection*{Persisting domains}
Using a \href{http://freemarker.org/}{Freemarker} template and a lot of string manipulation it is persisted into PDDL.
For parsing, we assume a correct and valid problem and use a formal grammar written in \href{http://www.antlr.org/}{ANTLR}
to parse PDDL into a generated code structure provided by ANTLR and the \texttt{maven-antlr-plugin}. The same grammar as for
problems is used.

TransportEditor doesn't load the PDDL domain definitions directly -- those are already built-in.
We only read the domain files to check which subset of conditions the user has chosen to model.

In the UI, we can also create a domain using a popup dialog backed by a \texttt{VariableDomainBuilder} in the background.
It is essentially switchboard for gathering the appropriate flags and other properties the domain should have.

\subsection*{Session}
The session is where everything comes together. It keeps an instance of the domain, problem, and plan (and planner and
validator, \ldots). We can use it to reason about what actions can be executed in the UI with the currently loaded
objects and also as a quick persistence solution -- if you save a session, you can then load it next time and
do not have to open all the individual parts again.

\subsubsection*{Visualizing sessions}

Sessions are visualized by visualizing all their (possible) parts.

\subsubsection*{Persisting sessions}
Sessions are persisted automatically to XML using \href{https://x-stream.github.io/}{XStream}. This means, all its properties
should be reasonably serializable (by reflection).

\subsection*{Planning}
Any class implementing the \texttt{Planner} interface can be set as the planner for a session and if it has all the properties
that are needed (domain \& problem), we can generate a plan using an instance of that class. TransportEditor supports
external (executable) planners out of the box, given that the executable adheres to a few rules (for details, see
\texttt{ExternalPlanner}). An end of planning event is raised after planning finishes, for UI redrawing purposes.

\subsection*{Validation}
Any class implementing the \texttt{Validator} interface can be used as a validator for plans in a planning session.
Validation happens automatically after planning in a session or it can be triggered manually. There are different
validators with different strictness (used for different domain variants). Choosing a wrong combination of domain,
problem and validator might cause false positives or false negatives, make sure to read the documentation of the
individual validators. TransportEditor supports a popular external validator called VAL, out of the box.

\subsection*{General notes}
There are few other small features of the project worth mentioning.

\subsubsection*{CDI \& the EventBus}
CDI (Context and Dependency Injection) using \href{http://weld.cdi-spec.org/}{Weld} is used for inversion of control
and for communication without tight coupling. Should only be used in the UI part of the project.

For event-driven communication on the front end, Guava's \texttt{EventBus} is used. Again, it enables persistent
reactive handling without tight coupling.

\subsubsection*{JavaFX properties and bindings}
The JavaFX based UI makes heavy use of bindings and properties, essential features of JavaFX. They enable
reactive changes to the UI in an efficient manner but can be a bit tricky when reading code that uses them.
For even more power, we use the \href{http://fxexperience.com/controlsfx/}{ControlsFX} library, but try to avoid it,
if possible.

\subsubsection*{Model immutability}
The model (mainly the package \texttt{com.oskopek.transporteditor.model}) is designed to be immutable
(excluding a few exceptions). This makes it easier to reason about complex, possibly multithreaded operations
on top of it. This note is useful to keep in mind when reading code that changes the model data.

\subsubsection*{Tests}
The project aims to be well tested and verified. To stick to these goals, we have several levels of tests,
that are run by a CI (Continuous Integration) system after every push and should also be run by developers
(at least) after every commit. The displayed test coverage in the README is calculated as an average of unit
and integration test coverage.

TransportEditor currently has 3 types of tests:

\begin{itemize}
\item Unit tests (\texttt{*Test.java}) -- simple and quick to run tests that test one thing and test it well.

\item Integration tests (\texttt{*IT.java}) -- complex tests that handle multiple moving parts at once. Usually involving IO or
other not easily mockable things. Try to avoid abusively writing these in favor of unit tests, if possible.

\item User interface tests (\texttt{*UI.java}) -- test the UI using \href{https://github.com/TestFX/TestFX}{TestFX}.
At the moment, they are under-represented and not run very often. Currently, CI doesn't run them by default.
\end{itemize}

\subsubsection*{Comments, code style}
TransportEditor employs a rigorous code style checker called \texttt{checkstyle} that is run automatically at every build.
Please adhere to that style when extending/editing the code base. Multiple other unwritten and unspecified rules might
apply. Please, do not take any style comments personally -- they are in place so that the code remains intact and
readable in the long term.

As part of the \texttt{checkstyle} process, JavaDoc comments are enforced on every method and class (excluding tests).
They should briefly describe the design/implementation choices, \textbf{why} they were made and any useful examples and or
other quirks.
%%%%%%%%% ENDREADME

