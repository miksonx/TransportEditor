\chapter{Transport domain problem analysis}

\section{Problem instances}

\TODO{Datasets}

\section{Problem features}

\section{TransportEditor -- a Transport domain planning environment}

TransportEditor aims to be a problem editor and plan visualizer for the Transport domain.
It is an intuitive GUI desktop application for making quick changes and re-planning, but also designing a new problem dataset from scratch. TransportEditor will help researchers working on this domain fine-tune their planners; they can visualize the various corner cases their planner fails to handle, step through the generated plan and find the points where their approach fails.
A secondary motivation is to be able to test approaches for creating plans for the domain.

The basic workflow of TransportEditor consists of the following user's steps:
\begin{itemize}
\item Select which formulation of the Transport domain they want to work with or create their own variant.
\item Load or create their own problem of the given domain. See section \ref{input-output} for details on the input format.
\item TransportEditor draws the given graph as good as it can.
\item Iterate among the following options:
\begin{itemize}
\item Load a planner executable and let TransportEditor run the planner on the loaded problem instance for a given time, then load the resulting plan.
\item Load a pre-generated plan.
\item Step through the individual plan actions and let TransportEditor visualize them.
The user can go forward and backward in the plan and inspect each step in great detail.
\item Edit the graph: add/remove/edit the location or properties of vehicles, packages, roads, locations and possibly petrol stations.
\item Save the currently generated plan.
\item Save the problem (along with the graph drawing hints).
\item Save the domain (export to a PDDL file).
\end{itemize}
\item Save and close the currently loaded problem. Exit the application or go back to the first step.
\end{itemize}

TransportEditor is a part of this thesis and you can find it on the attachment DVD, in the folder \TODO{releases}. If you want to find out more details about it, see the attached TransportEditor User Manual\ref{transporteditor-user-manual},
the TransportEditor Developer Documentation\ref{transporteditor-developer-documentation}
and the TransportEditor Developer JavaDoc\ref{transporteditor-developer-javadoc}.



