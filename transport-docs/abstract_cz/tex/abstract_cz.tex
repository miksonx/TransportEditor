\title{Pl{\'{a}}nov{\'{a}}n{\'{i}} pro p{\v{r}}epravn{\'{i}} probl{\'{e}}my}
\author{Ondrej~{\v{S}}kopek}
\date{\today}

%% Settings for single-side (simplex) printing
% Margins: left 40mm, right 25mm, top and bottom 25mm
% (but beware, LaTeX adds 1in implicitly)
\documentclass[12pt,a4paper]{article}
\setlength\textwidth{145mm}
\setlength\textheight{247mm}
\setlength\oddsidemargin{15mm}
\setlength\evensidemargin{15mm}
\setlength\topmargin{0mm}
\setlength\headsep{0mm}
\setlength\headheight{0mm}
% \openright makes the following text appear on a right-hand page
\let\openright=\clearpage

%% Settings for two-sided (duplex) printing
% \documentclass[12pt,a4paper,twoside,openright]{report}
% \setlength\textwidth{145mm}
% \setlength\textheight{247mm}
% \setlength\oddsidemargin{14.2mm}
% \setlength\evensidemargin{0mm}
% \setlength\topmargin{0mm}
% \setlength\headsep{0mm}
% \setlength\headheight{0mm}
% \let\openright=\cleardoublepage

%% Generate PDF/A-2u
\usepackage[a-2u]{pdfx}

%% Character encoding: usually latin2, cp1250 or utf8:
\usepackage[utf8]{inputenc}

%% Přepneme na českou sazbu a fonty Latin Modern
\usepackage[czech]{babel}
\usepackage{lmodern}
\usepackage[T1]{fontenc}
\usepackage{textcomp}

\begin{document}
\maketitle

Obrovské množství zdrojů je každodenně zbytečně promarněno kvůli neefektivnímu plánování přepravy.
Pomocí technik automatizovaného plánování navrhujeme několik plánovacích systémů pro efektivní řešení zjednodušených variant logistických problémů.
V těchto problémech jsou balíky doručovány do jejich cílů
pomocí nákladních vozidel pohybujících se na orientovaném, kladně ohodnoceném grafu, který představuje silniční síť.
Experimenty provedené na základě původních údajů z plánovacích soutěží ukazují, že naše přístupy dokáží zlepšit kvalitu řešení ve srovnání s plánovači nezávislými na plánovací doméně.
V neposlední řadě jsme vyvinuli nástroj TransportEditor, vizualizátor a editor těchto problémů, který umožňuje efektivní analýzu problémů, konstrukci plánovačů a introspekci plánů.

\end{document}
