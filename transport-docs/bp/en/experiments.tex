\chapter{Experimental evaluation}\label{experiments}

In this chapter, we will describe and run experiments
that compare our planners from the last two chapters
with state-of-the-art domain-independent planners from the IPC.
We will briefly discuss the acquired results and interpret them.

\section{Methodology}

Using our benchmarking software (described in the attachments), we will now run experiments in an
environment as similar to the original
IPC as possible, following almost all of the official rules.\puncfootnote{\url{http://icaps-conference.org/ipc2008/deterministic/CompetitionRules.html}}
All planners will be single-threaded and use a maximum of 2GB of memory, with a maximum run time of 30 minutes (planners will
get canceled and prompted for a plan at that time point).
Our planners explicitly and intentionally break the ``domain independence'' rule of the IPC.

The evaluation criteria remain the same as in the IPC:
we focus on plan \textit{quality} in favor of planner run time,
although we will mention runtimes.
The quality of a plan for a specific planner and sequential problem $p$ is defined as
$$\frac{\mt{total-cost}(\mt{planner}(p))}{\mt{total-cost}(BEST)},$$
where the results called $BEST$
are either precalculated outside of the competition environment or they are the best result of one of the planners in the competition, depending on which plan has a lower total cost.
Quality is, therefore, a number between $0$ and $1$.
The overall goal for planners is to maximize the sum of qualities over the problem instances in a given dataset, called the \textit{total quality}.
For temporal domains, quality is calculated in the same way, just by substituting total cost
for total time. We sometimes refer to total cost and total time as the \textit{score} of the planner, a term that is not dependent on the domain variant.

We will use four datasets for our experiments --- the seq-sat-6, seq-sat-7,
and seq-sat-8 datasets for sequential
and the tempo-sat-6 for temporal planners (Section~\ref{datasets}).
All the datasets used (and more) are available in the
software project sources (see the attached \nameref{cd-contents}).
Descriptions of planners that we will refer to by their competition names can be found in the respective competition results or booklets for IPC~2008,\puncfootnote{\url{http://icaps-conference.org/ipc2008/deterministic/Planners.html}} IPC~2011 \citep{Garcia-Olaya2011}, and IPC~2014 \citep{Vallati2015}.

In all planners where nondeterminism occurs,
we set the initial random seed to 2017
(on all individual problem runs).
All the following experiments
were run on a computer
with an 8 core 64-bit processor \texttt{Intel(R) Core(TM) i7-6700 CPU @ 3.40GHz}
and 16 GB of memory, running Gentoo Linux.
A big thank you goes out to the faculty for providing access to these machines.
We run all Java programs on Oracle's OpenJDK 
version \texttt{1.8.0\_121}, build \texttt{b13}.
The results presented here were obtained with the \TEver{} version of the TransportEditor project.\puncfootnote{Git tag \TEtag{}, available at \url{https://github.com/oskopek/TransportEditor}} The \texttt{NOTICE.txt} files
in the project module directories specify
the exact versions of libraries used.



















\section{Sequential Transport}

In this section, we present the results of our sequential planners on the seq-sat-6, seq-sat-7, and seq-sat-8 datasets. Specifically, these planners are included in the experiment:
\begin{description}
\item[MSFA3] The meta-heuristically weighted SFA planner (Section~\ref{msfa}) with the package and vehicle distance heuristic (Section~\ref{sfa3})
\item[MSFA5] The meta-heuristically weighted SFA planner with the general marking heuristic (Section~\ref{sfa5})
\item[RRAPN] The Randomized Restart Around Path Nearby planner (Section~\ref{rrapn})
\end{description}

\subsection{Results}\label{sequential-results}

We show an IPC quality table and a quality plot
for the experimental runs on the seq-sat-6 dataset (Figure~\ref{fig:seq-sat-6-results}), seq-sat-7 dataset (Figure~\ref{fig:seq-sat-7-results}), and the seq-sat-8 dataset (Figure~\ref{fig:seq-sat-8-results}). Details about the specific plans along with the benchmark results can be found in the \nameref{cd-contents}.

\begin{figure}[tbp]
\centering
\begin{subtable}{\textwidth}
\centering
\scriptsize
\renewcommand{\footnotesize}{\scriptsize}
\begin{tabular}{|l|rrrrrr|r|}
\hline
\textbf{\#} & \textbf{MSFA3} & \textbf{MSFA5} & \textbf{RRAPN} & \textbf{dtg-plan} & \textbf{lama} & \textbf{sgplan6} & \textbf{BEST}\\
\hline
p01 & {\footnotesize 54} \textbf{1.00} & {\footnotesize 54} \textbf{1.00} & {\footnotesize 54} \textbf{1.00} & {\footnotesize 54} \textbf{1.00} & {\footnotesize 54} \textbf{1.00} & {\footnotesize 54} \textbf{1.00} & 54\\
p02 & {\footnotesize 270} \textbf{1.00} & {\footnotesize 270} \textbf{1.00} & {\footnotesize 288} \textbf{0.94} & {\footnotesize 304} \textbf{0.89} & {\footnotesize 270} \textbf{1.00} & {\footnotesize 414} \textbf{0.65} & 270\\
p03 & {\footnotesize 409} \textbf{0.87} & {\footnotesize 355} \textbf{1.00} & {\footnotesize 419} \textbf{0.85} & {\footnotesize 635} \textbf{0.56} & {\footnotesize 497} \textbf{0.71} & {\footnotesize 801} \textbf{0.44} & 355\\
p04 & {\footnotesize 464} \textbf{0.78} & {\footnotesize 490} \textbf{0.74} & {\footnotesize 412} \textbf{0.88} & {\footnotesize 983} \textbf{0.37} & {\footnotesize 504} \textbf{0.72} & {\footnotesize 942} \textbf{0.39} & 363\\
p05 & {\footnotesize 704} \textbf{0.83} & {\footnotesize 704} \textbf{0.83} & {\footnotesize 582} \textbf{1.00} & {\footnotesize 1187} \textbf{0.49} & {\footnotesize 737} \textbf{0.79} & {\footnotesize 1186} \textbf{0.49} & 582\\
p06 & {\footnotesize 989} \textbf{0.76} & {\footnotesize 967} \textbf{0.78} & {\footnotesize 1026} \textbf{0.74} & {\footnotesize 1766} \textbf{0.43} & {\footnotesize 1117} \textbf{0.68} & {\footnotesize 1868} \textbf{0.40} & 755\\
p07 & {\footnotesize 1011} \textbf{0.98} & {\footnotesize 1011} \textbf{0.98} & {\footnotesize 988} \textbf{1.00} & {\footnotesize 1868} \textbf{0.53} & {\footnotesize 1260} \textbf{0.78} & {\footnotesize 2081} \textbf{0.47} & 988\\
p08 & {\footnotesize 1053} \textbf{0.88} & {\footnotesize 1053} \textbf{0.88} & {\footnotesize 925} \textbf{1.00} & {\footnotesize 2166} \textbf{0.43} & {\footnotesize 1216} \textbf{0.76} & {\footnotesize 2135} \textbf{0.43} & 925\\
p09 & {\footnotesize 1027} \textbf{0.93} & {\footnotesize 1027} \textbf{0.93} & {\footnotesize 955} \textbf{1.00} & {\footnotesize 1880} \textbf{0.51} & {\footnotesize 1001} \textbf{0.95} & {\footnotesize 2143} \textbf{0.45} & 955\\
p10 & {\footnotesize 1360} \textbf{0.78} & {\footnotesize 1360} \textbf{0.78} & {\footnotesize 1059} \textbf{1.00} & {\footnotesize 2260} \textbf{0.47} & {\footnotesize 1285} \textbf{0.82} & {\footnotesize 2091} \textbf{0.51} & 1059\\
p11 & {\footnotesize 473} \textbf{1.00} & {\footnotesize 473} \textbf{1.00} & {\footnotesize 473} \textbf{1.00} & {\footnotesize 473} \textbf{1.00} & {\footnotesize 473} \textbf{1.00} & {\footnotesize 475} \textbf{1.00} & 473\\
p12 & {\footnotesize 823} \textbf{0.97} & {\footnotesize 823} \textbf{0.97} & {\footnotesize 872} \textbf{0.91} & {\footnotesize 800} \textbf{0.99} & {\footnotesize 795} \textbf{1.00} & {\footnotesize 1244} \textbf{0.64} & 795\\
p13 & {\footnotesize 1096} \textbf{0.88} & {\footnotesize 1096} \textbf{0.88} & {\footnotesize 965} \textbf{1.00} & {\footnotesize 2751} \textbf{0.35} & {\footnotesize 1147} \textbf{0.84} & {\footnotesize 2827} \textbf{0.34} & 965\\
p14 & {\footnotesize 1582} \textbf{1.00} & {\footnotesize 1582} \textbf{1.00} & {\footnotesize 1966} \textbf{0.80} & {\footnotesize 3507} \textbf{0.45} & {\footnotesize 2157} \textbf{0.73} & {\footnotesize 3328} \textbf{0.48} & 1582\\
p15 & {\footnotesize 2367} \textbf{0.96} & {\footnotesize 2280} \textbf{1.00} & {\footnotesize 3129} \textbf{0.73} & {\footnotesize 5221} \textbf{0.44} & {\footnotesize 2954} \textbf{0.77} & {\footnotesize 5659} \textbf{0.40} & 2280\\
p16 & {\footnotesize 2321} \textbf{1.00} & {\footnotesize 2321} \textbf{1.00} & {\footnotesize 2765} \textbf{0.84} & {\footnotesize 6199} \textbf{0.37} & {\footnotesize 4928} \textbf{0.47} & {\footnotesize 6144} \textbf{0.38} & 2321\\
p17 & {\footnotesize 3209} \textbf{1.00} & {\footnotesize 3209} \textbf{1.00} & {\footnotesize 4321} \textbf{0.74} & {\footnotesize 7239} \textbf{0.44} & {\footnotesize 4193} \textbf{0.77} & {\footnotesize 7494} \textbf{0.43} & 3209\\
p18 & {\footnotesize 3322} \textbf{0.88} & {\footnotesize 2936} \textbf{1.00} & {\footnotesize 3663} \textbf{0.80} & {\footnotesize 7542} \textbf{0.39} & {\footnotesize 4151} \textbf{0.71} & {\footnotesize 7737} \textbf{0.38} & 2936\\
p19 & {\footnotesize 5051} \textbf{1.00} & {\footnotesize 5051} \textbf{1.00} & {\footnotesize 5073} \textbf{1.00} & {\footnotesize 9921} \textbf{0.51} & {\footnotesize 7648} \textbf{0.66} & {\footnotesize 8991} \textbf{0.56} & 5051\\
p20 & {\footnotesize 3636} \textbf{1.00} & {\footnotesize 3873} \textbf{0.94} & {\footnotesize 4674} \textbf{0.78} & uns. & {\footnotesize 6773} \textbf{0.54} & {\footnotesize 8663} \textbf{0.42} & 3636\\
p21 & {\footnotesize 431} \textbf{1.00} & {\footnotesize 431} \textbf{1.00} & {\footnotesize 431} \textbf{1.00} & {\footnotesize 431} \textbf{1.00} & {\footnotesize 431} \textbf{1.00} & {\footnotesize 431} \textbf{1.00} & 431\\
p22 & {\footnotesize 675} \textbf{1.00} & {\footnotesize 675} \textbf{1.00} & {\footnotesize 677} \textbf{1.00} & {\footnotesize 679} \textbf{0.99} & {\footnotesize 675} \textbf{1.00} & {\footnotesize 1268} \textbf{0.53} & 675\\
p23 & {\footnotesize 1140} \textbf{0.73} & {\footnotesize 1140} \textbf{0.73} & {\footnotesize 897} \textbf{0.93} & {\footnotesize 2414} \textbf{0.35} & {\footnotesize 837} \textbf{1.00} & {\footnotesize 2119} \textbf{0.39} & 837\\
p24 & {\footnotesize 1227} \textbf{1.00} & {\footnotesize 1227} \textbf{1.00} & {\footnotesize 1352} \textbf{0.91} & {\footnotesize 2790} \textbf{0.44} & {\footnotesize 1301} \textbf{0.94} & {\footnotesize 2909} \textbf{0.42} & 1227\\
p25 & {\footnotesize 1943} \textbf{0.92} & {\footnotesize 1943} \textbf{0.92} & {\footnotesize 1785} \textbf{1.00} & {\footnotesize 4007} \textbf{0.45} & {\footnotesize 1833} \textbf{0.97} & {\footnotesize 3764} \textbf{0.47} & 1785\\
p26 & {\footnotesize 2421} \textbf{0.74} & {\footnotesize 2421} \textbf{0.74} & {\footnotesize 1797} \textbf{1.00} & {\footnotesize 4036} \textbf{0.45} & {\footnotesize 2502} \textbf{0.72} & {\footnotesize 3598} \textbf{0.50} & 1797\\
p27 & {\footnotesize 3255} \textbf{0.77} & {\footnotesize 3255} \textbf{0.77} & {\footnotesize 2521} \textbf{1.00} & {\footnotesize 5791} \textbf{0.44} & {\footnotesize 3317} \textbf{0.76} & {\footnotesize 5948} \textbf{0.42} & 2521\\
p28 & {\footnotesize 2465} \textbf{1.00} & {\footnotesize 2465} \textbf{1.00} & {\footnotesize 2594} \textbf{0.95} & {\footnotesize 6346} \textbf{0.39} & {\footnotesize 3027} \textbf{0.81} & {\footnotesize 7300} \textbf{0.34} & 2465\\
p29 & {\footnotesize 2817} \textbf{1.00} & {\footnotesize 2890} \textbf{0.97} & {\footnotesize 2871} \textbf{0.98} & {\footnotesize 7168} \textbf{0.39} & {\footnotesize 3294} \textbf{0.86} & {\footnotesize 7237} \textbf{0.39} & 2817\\
p30 & {\footnotesize 4703} \textbf{0.76} & {\footnotesize 4703} \textbf{0.76} & {\footnotesize 3595} \textbf{1.00} & uns. & {\footnotesize 5513} \textbf{0.65} & {\footnotesize 7892} \textbf{0.46} & 3595\\
\hline
\textbf{total} & \textbf{27.43} & \textbf{27.61} & \textbf{27.77} & \textbf{15.50} & \textbf{24.43} & \textbf{15.19} & \\
\hline
\end{tabular}


\caption{Quality and score of sequential planners on the seq-sat-6 dataset.}
\label{tab:seq-sat-6-ipc-scores}
\end{subtable}

\vspace{0.5cm}
\begin{subfigure}{\textwidth}
\centering
\includegraphics[width=1.0\textwidth]{../imga/seq-sat-6-quality}
\caption{Quality plot of sequential planners on the seq-sat-6 dataset.}
\label{fig:seq-sat-6-quality}
\end{subfigure}
\caption{Planner results on seq-sat-6.}
\label{fig:seq-sat-6-results}
\end{figure}

\begin{figure}[tbp]
\centering
\begin{subtable}{\textwidth}
\centering
\scriptsize
\renewcommand{\footnotesize}{\scriptsize}
\begin{tabular}{|l|rrrrrrr|r|}
\hline
\textbf{Problem} & \textbf{MSFA3} & \textbf{RRAPN} & \textbf{SFA3} & \textbf{dae\_yahsp} & \textbf{lama-2008} & \textbf{lama-2011} & \textbf{roamer} & \textbf{BEST}\\
\hline
p01 & uns. & {\footnotesize 915} \textbf{1.00} & uns. & {\footnotesize 1498} \textbf{0.61} & {\footnotesize 1050} \textbf{0.87} & {\footnotesize 1485} \textbf{0.62} & {\footnotesize 1050} \textbf{0.87} & 915\\
p02 & {\footnotesize 992} \textbf{0.94} & {\footnotesize 936} \textbf{1.00} & uns. & {\footnotesize 1701} \textbf{0.55} & {\footnotesize 996} \textbf{0.94} & {\footnotesize 1010} \textbf{0.93} & {\footnotesize 996} \textbf{0.94} & 936\\
p03 & uns. & uns. & uns. & {\footnotesize 4971} \textbf{0.66} & {\footnotesize 3313} \textbf{0.99} & {\footnotesize 3882} \textbf{0.84} & {\footnotesize 3275} \textbf{1.00} & 3275\\
p04 & uns. & {\footnotesize 2768} \textbf{0.95} & uns. & {\footnotesize 2618} \textbf{1.00} & {\footnotesize 5135} \textbf{0.51} & {\footnotesize 3741} \textbf{0.70} & {\footnotesize 5841} \textbf{0.45} & 2618\\
p05 & uns. & {\footnotesize 4331} \textbf{0.85} & uns. & {\footnotesize 3661} \textbf{1.00} & {\footnotesize 5481} \textbf{0.67} & {\footnotesize 4805} \textbf{0.76} & {\footnotesize 5553} \textbf{0.66} & 3661\\
p06 & uns. & {\footnotesize 3703} \textbf{0.92} & uns. & {\footnotesize 3974} \textbf{0.86} & {\footnotesize 4320} \textbf{0.79} & {\footnotesize 5415} \textbf{0.63} & {\footnotesize 4681} \textbf{0.73} & 3401\\
p07 & uns. & {\footnotesize 5040} \textbf{1.00} & uns. & {\footnotesize 5683} \textbf{0.89} & {\footnotesize 6652} \textbf{0.76} & {\footnotesize 7222} \textbf{0.70} & {\footnotesize 7403} \textbf{0.68} & 5040\\
p08 & uns. & {\footnotesize 1051} \textbf{1.00} & uns. & {\footnotesize 2067} \textbf{0.51} & {\footnotesize 1211} \textbf{0.87} & {\footnotesize 1452} \textbf{0.72} & {\footnotesize 1211} \textbf{0.87} & 1051\\
p09 & uns. & {\footnotesize 4706} \textbf{1.00} & uns. & {\footnotesize 5863} \textbf{0.80} & {\footnotesize 6786} \textbf{0.69} & {\footnotesize 6479} \textbf{0.73} & {\footnotesize 6806} \textbf{0.69} & 4706\\
p10 & uns. & {\footnotesize 3686} \textbf{1.00} & uns. & {\footnotesize 6145} \textbf{0.60} & {\footnotesize 5943} \textbf{0.62} & {\footnotesize 5641} \textbf{0.65} & {\footnotesize 5445} \textbf{0.68} & 3686\\
p11 & uns. & {\footnotesize 1222} \textbf{1.00} & uns. & {\footnotesize 2139} \textbf{0.57} & {\footnotesize 1547} \textbf{0.79} & {\footnotesize 2113} \textbf{0.58} & {\footnotesize 1901} \textbf{0.64} & 1222\\
p12 & uns. & {\footnotesize 1408} \textbf{1.00} & uns. & {\footnotesize 2577} \textbf{0.55} & {\footnotesize 1929} \textbf{0.73} & {\footnotesize 1947} \textbf{0.72} & {\footnotesize 1915} \textbf{0.74} & 1408\\
p13 & uns. & {\footnotesize 1730} \textbf{1.00} & uns. & {\footnotesize 2860} \textbf{0.60} & uns. & {\footnotesize 2932} \textbf{0.59} & {\footnotesize 2746} \textbf{0.63} & 1730\\
p14 & uns. & {\footnotesize 6290} \textbf{1.00} & uns. & {\footnotesize 9004} \textbf{0.70} & {\footnotesize 7925} \textbf{0.79} & {\footnotesize 8493} \textbf{0.74} & {\footnotesize 7940} \textbf{0.79} & 6290\\
p15 & uns. & {\footnotesize 6070} \textbf{1.00} & uns. & {\footnotesize 7209} \textbf{0.84} & {\footnotesize 7192} \textbf{0.84} & {\footnotesize 6909} \textbf{0.88} & {\footnotesize 6924} \textbf{0.88} & 6070\\
p16 & uns. & {\footnotesize 5745} \textbf{1.00} & uns. & {\footnotesize 7533} \textbf{0.76} & {\footnotesize 6951} \textbf{0.83} & uns. & uns. & 5745\\
p17 & uns. & {\footnotesize 4929} \textbf{1.00} & uns. & {\footnotesize 9466} \textbf{0.52} & {\footnotesize 6166} \textbf{0.80} & {\footnotesize 5899} \textbf{0.84} & {\footnotesize 5209} \textbf{0.95} & 4929\\
p18 & uns. & {\footnotesize 3666} \textbf{1.00} & uns. & {\footnotesize 6659} \textbf{0.55} & {\footnotesize 5381} \textbf{0.68} & {\footnotesize 5690} \textbf{0.64} & {\footnotesize 3902} \textbf{0.94} & 3666\\
p19 & uns. & {\footnotesize 4277} \textbf{1.00} & uns. & {\footnotesize 8011} \textbf{0.53} & {\footnotesize 5716} \textbf{0.75} & {\footnotesize 5777} \textbf{0.74} & {\footnotesize 5257} \textbf{0.81} & 4277\\
p20 & uns. & {\footnotesize 4174} \textbf{1.00} & uns. & {\footnotesize 6891} \textbf{0.61} & {\footnotesize 5831} \textbf{0.72} & {\footnotesize 4435} \textbf{0.94} & {\footnotesize 4793} \textbf{0.87} & 4174\\
\hline
\textbf{total} & \textbf{0.94} & \textbf{18.71} & \textbf{0.00} & \textbf{13.71} & \textbf{14.63} & \textbf{13.95} & \textbf{14.81} & \\
\hline
\end{tabular}


\caption{Quality and score of sequential planners on the seq-sat-7 dataset.}
\label{tab:seq-sat-7-ipc-scores}
\end{subtable}

\vspace{0.5cm}
\begin{subfigure}{\textwidth}
\centering
\includegraphics[width=1.0\textwidth]{../imga/seq-sat-7-quality}
\caption{Quality plot of sequential planners on the seq-sat-7 dataset.}
\label{fig:seq-sat-7-quality}
\end{subfigure}
\caption{Planner results on seq-sat-7.}
\label{fig:seq-sat-7-results}
\end{figure}

\begin{figure}[tbp]
\centering
\begin{subtable}{\textwidth}
\centering
\scriptsize
\renewcommand{\footnotesize}{\scriptsize}
\begin{tabular}{|l|rrrrrr|r|}
\hline
\textbf{\#} & \textbf{MSFA3} & \textbf{MSFA5} & \textbf{RRAPN} & \textbf{ibacop} & \textbf{mercury} & \textbf{yahsp3-mt} & \textbf{BEST}\\
\hline
p01 & {\footnotesize 1596} \textbf{0.82} & {\footnotesize 1596} \textbf{0.82} & {\footnotesize 1600} \textbf{0.82} & {\footnotesize 2045} \textbf{0.64} & {\footnotesize 1309} \textbf{1.00} & {\footnotesize 3044} \textbf{0.43} & 1309\\
p02 & {\footnotesize 2109} \textbf{1.00} & {\footnotesize 2109} \textbf{1.00} & {\footnotesize 2300} \textbf{0.92} & {\footnotesize 5902} \textbf{0.36} & {\footnotesize 2125} \textbf{0.99} & {\footnotesize 4250} \textbf{0.50} & 2109\\
p03 & {\footnotesize 1879} \textbf{0.82} & {\footnotesize 1879} \textbf{0.82} & {\footnotesize 1785} \textbf{0.86} & {\footnotesize 2653} \textbf{0.58} & {\footnotesize 1539} \textbf{1.00} & {\footnotesize 3274} \textbf{0.47} & 1539\\
p04 & {\footnotesize 5163} \textbf{0.99} & {\footnotesize 5092} \textbf{1.00} & {\footnotesize 7508} \textbf{0.68} & {\footnotesize 8871} \textbf{0.57} & {\footnotesize 5678} \textbf{0.90} & {\footnotesize 8228} \textbf{0.62} & 5092\\
p05 & {\footnotesize 5394} \textbf{1.00} & {\footnotesize 5708} \textbf{0.94} & {\footnotesize 7518} \textbf{0.72} & {\footnotesize 14170} \textbf{0.38} & {\footnotesize 6235} \textbf{0.87} & {\footnotesize 10938} \textbf{0.49} & 5394\\
p06 & {\footnotesize 5163} \textbf{0.99} & {\footnotesize 5092} \textbf{1.00} & {\footnotesize 7508} \textbf{0.68} & {\footnotesize 8871} \textbf{0.57} & {\footnotesize 5678} \textbf{0.90} & {\footnotesize 8228} \textbf{0.62} & 5092\\
p07 & {\footnotesize 4202} \textbf{1.00} & {\footnotesize 4202} \textbf{1.00} & {\footnotesize 4996} \textbf{0.84} & {\footnotesize 11802} \textbf{0.36} & {\footnotesize 4839} \textbf{0.87} & {\footnotesize 7804} \textbf{0.54} & 4202\\
p08 & {\footnotesize 4996} \textbf{0.89} & {\footnotesize 4948} \textbf{0.90} & {\footnotesize 5276} \textbf{0.85} & {\footnotesize 12762} \textbf{0.35} & {\footnotesize 4467} \textbf{1.00} & {\footnotesize 8590} \textbf{0.52} & 4467\\
p09 & {\footnotesize 4202} \textbf{1.00} & {\footnotesize 4202} \textbf{1.00} & {\footnotesize 4996} \textbf{0.84} & {\footnotesize 11802} \textbf{0.36} & {\footnotesize 4839} \textbf{0.87} & {\footnotesize 7680} \textbf{0.55} & 4202\\
p10 & {\footnotesize 4473} \textbf{1.00} & {\footnotesize 4473} \textbf{1.00} & {\footnotesize 5731} \textbf{0.78} & {\footnotesize 8260} \textbf{0.54} & {\footnotesize 4626} \textbf{0.97} & {\footnotesize 8410} \textbf{0.53} & 4473\\
p11 & {\footnotesize 1395} \textbf{0.96} & {\footnotesize 1395} \textbf{0.96} & {\footnotesize 1555} \textbf{0.86} & {\footnotesize 2154} \textbf{0.62} & {\footnotesize 1336} \textbf{1.00} & {\footnotesize 2429} \textbf{0.55} & 1336\\
p12 & {\footnotesize 1579} \textbf{1.00} & {\footnotesize 1579} \textbf{1.00} & {\footnotesize 2073} \textbf{0.76} & {\footnotesize 2524} \textbf{0.63} & {\footnotesize 1641} \textbf{0.96} & {\footnotesize 3646} \textbf{0.43} & 1579\\
p13 & {\footnotesize 1683} \textbf{0.68} & {\footnotesize 1683} \textbf{0.68} & {\footnotesize 1537} \textbf{0.75} & {\footnotesize 2085} \textbf{0.55} & {\footnotesize 1147} \textbf{1.00} & {\footnotesize 3700} \textbf{0.31} & 1147\\
p14 & {\footnotesize 7196} \textbf{0.83} & {\footnotesize 7196} \textbf{0.83} & {\footnotesize 6764} \textbf{0.88} & {\footnotesize 10667} \textbf{0.56} & {\footnotesize 5974} \textbf{1.00} & {\footnotesize 9334} \textbf{0.64} & 5974\\
p15 & {\footnotesize 7671} \textbf{0.69} & {\footnotesize 7671} \textbf{0.69} & {\footnotesize 7906} \textbf{0.67} & {\footnotesize 12975} \textbf{0.41} & {\footnotesize 5320} \textbf{1.00} & {\footnotesize 11822} \textbf{0.45} & 5320\\
p16 & {\footnotesize 5179} \textbf{0.91} & {\footnotesize 5107} \textbf{0.92} & {\footnotesize 6914} \textbf{0.68} & {\footnotesize 10918} \textbf{0.43} & {\footnotesize 4695} \textbf{1.00} & {\footnotesize 8536} \textbf{0.55} & 4695\\
p17 & {\footnotesize 4823} \textbf{0.94} & {\footnotesize 4646} \textbf{0.98} & {\footnotesize 5371} \textbf{0.85} & {\footnotesize 9659} \textbf{0.47} & {\footnotesize 4540} \textbf{1.00} & {\footnotesize 8107} \textbf{0.56} & 4540\\
p18 & {\footnotesize 4585} \textbf{1.00} & {\footnotesize 4585} \textbf{1.00} & {\footnotesize 5681} \textbf{0.81} & {\footnotesize 10755} \textbf{0.43} & {\footnotesize 4840} \textbf{0.95} & {\footnotesize 10521} \textbf{0.44} & 4585\\
p19 & {\footnotesize 3812} \textbf{1.00} & {\footnotesize 3812} \textbf{1.00} & {\footnotesize 4837} \textbf{0.79} & {\footnotesize 10780} \textbf{0.35} & {\footnotesize 3881} \textbf{0.98} & {\footnotesize 7322} \textbf{0.52} & 3812\\
p20 & {\footnotesize 4173} \textbf{0.92} & {\footnotesize 3923} \textbf{0.98} & {\footnotesize 4991} \textbf{0.77} & {\footnotesize 9632} \textbf{0.40} & {\footnotesize 3853} \textbf{1.00} & {\footnotesize 6643} \textbf{0.58} & 3853\\
\hline
\textbf{total} & \textbf{18.44} & \textbf{18.53} & \textbf{15.80} & \textbf{9.56} & \textbf{19.25} & \textbf{10.29} & \\
\hline
\end{tabular}


\caption{Quality and score of sequential planners on the seq-sat-8 dataset.}
\label{tab:seq-sat-8-ipc-scores}
\end{subtable}

\vspace{0.5cm}
\begin{subfigure}{\textwidth}
\centering
\includegraphics[width=1.0\textwidth]{../imga/seq-sat-8-quality}
\caption{Quality plot of sequential planners on the seq-sat-8 dataset.}
\label{fig:seq-sat-8-quality}
\end{subfigure}
\caption{Planner results on seq-sat-8.}
\label{fig:seq-sat-8-results}
\end{figure}



\subsubsection{IPC 2008}

In the updated results of the sequential satisficing track of IPC 2008\footnote{\url{http://icaps-conference.org/ipc2008/deterministic/Results.html}} published after the competition,
the overall winner LAMA (a Fast Downward based planner)
was hands-down the best planner on the sequential Transport domain, winning
with a total quality of $28.93/30$, where all other planners had less than $20/30$.
Only 5 plans generated by LAMA had a worse total cost than the best known plans.

After adding our planners to the results,
the total quality of LAMA drops to $24.43/30$,
because several larger problems were solved better than the
best known solution from IPC 2008.
Our best planner on the IPC 2008 dataset, RRAPN, achieves a total quality of $27.75/30$,
which is a slight improvement over LAMA and other planners. The biggest gain of RRAPN is in being able to approximate
solutions of larger problems fast, which can be observed on
the results on problems 7--10 or 25--27,
which are the largest problems.
On the other hand, RRAPN fails to achieve optimal scores
on some smaller problems like problem number 2 or 12,
due to its explicit nature.

MSFA3 and MSFA5 are quite similar both in their construction and results on this dataset.
In the final summary, MSFA5 comes out on top as the better one of the two planners.

All of our three planners beat all planners from the original competition based on total score,
which despite the possible minor differences in the computing environment indicates that
domain knowledge indeed helps us generate better plans.

\subsubsection{IPC 2011}

The 2011 competition featured 20 sequential Transport problems,
with 4 planners (dae\_yahsp, LAMA 2008 and 2011, and roamer) achieving a total quality of more than $15/20$.
Interestingly, LAMA 2008 was able to produce better plans than its 2011 version in 12 out of 20 problems. The overall winner on Transport in 2011, roamer, achieved comparable scores on most problems to both versions of LAMA. The fourth best planner, dae\_yahsp, performed consistenly worse by a small margin, except for problems number 4 and 5, where it beat our planner RRAPN.

RRAPN consistenly achieves better scores than all domain-independent planners from the original competition in 18 out of the 20 problems. This can again be attributed to the size
of the problems (see Table~\ref{tab:dataset-dimensions}).

Even though RRAPN is better than the original planners more often than both MSFA planners,
they are come out on top based on total cost. Even more interesting, the problems
where they perform well are complementary to the ones where RRAPN performs well,
as is visible on the results of problems 3--6, 10--12, and 13--15.

\subsubsection{IPC 2014}

In the satisficing track of IPC 2014, the winner on the Transport domain
was without a doubt the Mercury planner, achieving
a stunning $20/20$ total quality. Even more interesting is the fact that
the runner-up yahsp3-mt achieved a score of only $10.74/20$
and all other planners achieved sub $10/20$ total quality,
accentuating the performance of Mercury even more.

After adding the results of our planners to the quality table,
the total quality of yahsp3-mt is lowered to $10.29/20$.
Mercury loses its spotless results, but still significantly dominates all
other planners, including ours, at $19.25/20$.

RRAPN manages to outperform yahsp3-mt with $15.88/20$, yet it fails
to match the results of Mercury, not even in one problem.
Both MSFA planners outperform RRAPN on this dataset with qualities around $18.50/20$,
but still do not come reasonably close to beating Mercury.
However, they do (marginally) outperform Mercury on some problems, like
problems 4--7, 9--10, 12, and 18--19.
The results of MSFA3 and MSFA5 on this dataset are almost identical.

\TODO{fix seq-sat-8 quality plot for MSFA5}

















\section{Temporal Transport}

In this section, we present the results of our temporal planners on the tempo-sat-6 dataset. The following planners are included in the experiment:
\begin{description}
\item[AdHocTemp] The ? Temporal planner (Section~\ref{temporal-approach})
\item[MSFA5Sched] The scheduled MSFA5 planner (Sections~\ref{msfa} and~\ref{sfa5})
\item[RRAPNSched] The scheduled (Section~\ref{sched}) RRAPN planner (Section~\ref{rrapn})
\item[TFD2014] The Temporal Fast Downward planner, version 0.4 from IPC~2014 \citep[Preferring Preferred Operators in Temporal Fast Downward]{Vallati2015}
\end{description}

\subsection{Results}\label{temporal-results}

We show an IPC quality table and a quality plot of an experimental run of these planners on the tempo-sat-6 dataset (Figure~\ref{fig:tempo-sat-6-results}).
Additionally, sample Gantt charts \citep{Gantt1910} of two chosen plans are shown in Figure~\ref{fig:tempo-sat-6-gantt}.
The generated plans and benchmark results can be found in the attached \nameref{cd-contents}.

\begin{figure}[tbp]
\centering
\begin{subtable}{\textwidth}
\centering
\scriptsize
\renewcommand{\footnotesize}{\scriptsize}
\begin{tabular}{lrrrrrrr}
\toprule
\textbf{\#} & \textbf{MSFA5Sched} & \textbf{RRAPNSched} & \textbf{TFD2014} & \textbf{TRRAPN} & \textbf{sgplan6} & \textbf{tfd} & \textbf{BEST}\\
\midrule
\multicolumn{1}{l|}{p01} & {\footnotesize 52} \textbf{1.00} & {\footnotesize 52} \textbf{1.00} & {\footnotesize 52.02} \textbf{1.00} & {\footnotesize 52} \textbf{1.00} & {\footnotesize 52} \textbf{1.00} & {\footnotesize 52} \textbf{1.00} & \multicolumn{1}{|r}{52}\\
\multicolumn{1}{l|}{p02} & {\footnotesize 125.01} \textbf{0.98} & {\footnotesize 126.01} \textbf{0.98} & {\footnotesize 150.11} \textbf{0.82} & {\footnotesize 126.01} \textbf{0.98} & {\footnotesize 217} \textbf{0.57} & {\footnotesize 208} \textbf{0.59} & \multicolumn{1}{|r}{123}\\
\multicolumn{1}{l|}{p03} & {\footnotesize 252.02} \textbf{0.75} & {\footnotesize 198.01} \textbf{0.95} & {\footnotesize 252.14} \textbf{0.75} & {\footnotesize 198.01} \textbf{0.95} & {\footnotesize 432} \textbf{0.44} & {\footnotesize 669} \textbf{0.28} & \multicolumn{1}{|r}{189}\\
\multicolumn{1}{l|}{p04} & {\footnotesize 341.02} \textbf{0.76} & {\footnotesize 260.02} \textbf{1.00} & {\footnotesize 425.29} \textbf{0.61} & {\footnotesize 267.01} \textbf{0.97} & {\footnotesize 845} \textbf{0.31} & uns. & \multicolumn{1}{|r}{260.02}\\
\multicolumn{1}{l|}{p05} & {\footnotesize 285.03} \textbf{0.85} & {\footnotesize 243.02} \textbf{1.00} & {\footnotesize 367.32} \textbf{0.66} & {\footnotesize 249.02} \textbf{0.98} & {\footnotesize 359} \textbf{0.68} & uns. & \multicolumn{1}{|r}{243.02}\\
\multicolumn{1}{l|}{p06} & {\footnotesize 316.02} \textbf{0.80} & {\footnotesize 253.01} \textbf{1.00} & {\footnotesize 408.31} \textbf{0.62} & {\footnotesize 265.02} \textbf{0.95} & {\footnotesize 965} \textbf{0.26} & uns. & \multicolumn{1}{|r}{253.01}\\
\multicolumn{1}{l|}{p07} & uns. & {\footnotesize 367.03} \textbf{1.00} & uns. & {\footnotesize 369.03} \textbf{0.99} & uns. & uns. & \multicolumn{1}{|r}{367.03}\\
\multicolumn{1}{l|}{p08} & uns. & {\footnotesize 481.04} \textbf{1.00} & uns. & {\footnotesize 532.04} \textbf{0.90} & uns. & uns. & \multicolumn{1}{|r}{481.04}\\
\multicolumn{1}{l|}{p09} & uns. & {\footnotesize 286.03} \textbf{1.00} & {\footnotesize 494.44} \textbf{0.58} & {\footnotesize 309.03} \textbf{0.93} & uns. & uns. & \multicolumn{1}{|r}{286.03}\\
\multicolumn{1}{l|}{p10} & uns. & uns. & {\footnotesize 939.8} \textbf{0.88} & {\footnotesize 827.07} \textbf{1.00} & uns. & uns. & \multicolumn{1}{|r}{827.07}\\
\multicolumn{1}{l|}{p11} & {\footnotesize 332.01} \textbf{1.00} & {\footnotesize 332.01} \textbf{1.00} & {\footnotesize 342.09} \textbf{0.97} & {\footnotesize 332.01} \textbf{1.00} & {\footnotesize 629} \textbf{0.53} & {\footnotesize 549} \textbf{0.60} & \multicolumn{1}{|r}{332}\\
\multicolumn{1}{l|}{p12} & {\footnotesize 483.01} \textbf{0.90} & {\footnotesize 490.01} \textbf{0.88} & {\footnotesize 543.13} \textbf{0.80} & {\footnotesize 490.01} \textbf{0.88} & {\footnotesize 817} \textbf{0.53} & {\footnotesize 982} \textbf{0.44} & \multicolumn{1}{|r}{433}\\
\multicolumn{1}{l|}{p13} & {\footnotesize 572.02} \textbf{0.68} & {\footnotesize 459.01} \textbf{0.85} & {\footnotesize 1172.38} \textbf{0.33} & {\footnotesize 434.01} \textbf{0.90} & {\footnotesize 650} \textbf{0.60} & {\footnotesize 3383} \textbf{0.11} & \multicolumn{1}{|r}{389}\\
\multicolumn{1}{l|}{p14} & {\footnotesize 777.03} \textbf{0.77} & {\footnotesize 621.02} \textbf{0.96} & {\footnotesize 1938.75} \textbf{0.31} & {\footnotesize 620.02} \textbf{0.96} & uns. & uns. & \multicolumn{1}{|r}{595}\\
\multicolumn{1}{l|}{p15} & {\footnotesize 1081.04} \textbf{0.76} & {\footnotesize 866.04} \textbf{0.95} & {\footnotesize 1143.45} \textbf{0.72} & {\footnotesize 860.04} \textbf{0.96} & {\footnotesize 2249} \textbf{0.37} & uns. & \multicolumn{1}{|r}{824}\\
\multicolumn{1}{l|}{p16} & {\footnotesize 1532.07} \textbf{0.49} & {\footnotesize 760.03} \textbf{0.98} & {\footnotesize 2198.97} \textbf{0.34} & {\footnotesize 752.03} \textbf{0.99} & {\footnotesize 1875} \textbf{0.40} & uns. & \multicolumn{1}{|r}{748}\\
\multicolumn{1}{l|}{p17} & {\footnotesize 1317.07} \textbf{0.60} & {\footnotesize 906.03} \textbf{0.87} & {\footnotesize 2393.97} \textbf{0.33} & {\footnotesize 916.04} \textbf{0.86} & {\footnotesize 3331} \textbf{0.24} & uns. & \multicolumn{1}{|r}{789}\\
\multicolumn{1}{l|}{p18} & {\footnotesize 1960.09} \textbf{0.62} & {\footnotesize 1217.05} \textbf{1.00} & uns. & {\footnotesize 1224.06} \textbf{0.99} & uns. & uns. & \multicolumn{1}{|r}{1217.05}\\
\multicolumn{1}{l|}{p19} & {\footnotesize 2226.12} \textbf{0.56} & {\footnotesize 1266.06} \textbf{0.99} & uns. & {\footnotesize 1254.06} \textbf{1.00} & uns. & uns. & \multicolumn{1}{|r}{1254.06}\\
\multicolumn{1}{l|}{p20} & {\footnotesize 2596.13} \textbf{0.42} & {\footnotesize 1399.07} \textbf{0.77} & uns. & {\footnotesize 1488.08} \textbf{0.73} & {\footnotesize 6362} \textbf{0.17} & uns. & \multicolumn{1}{|r}{1084}\\
\multicolumn{1}{l|}{p21} & {\footnotesize 94.02} \textbf{0.67} & {\footnotesize 69.01} \textbf{0.91} & {\footnotesize 102.14} \textbf{0.62} & {\footnotesize 69.01} \textbf{0.91} & {\footnotesize 113} \textbf{0.56} & {\footnotesize 161} \textbf{0.39} & \multicolumn{1}{|r}{63}\\
\multicolumn{1}{l|}{p22} & {\footnotesize 192.03} \textbf{0.49} & {\footnotesize 114.01} \textbf{0.82} & {\footnotesize 265.38} \textbf{0.35} & {\footnotesize 114.01} \textbf{0.82} & {\footnotesize 238} \textbf{0.39} & uns. & \multicolumn{1}{|r}{94}\\
\multicolumn{1}{l|}{p23} & {\footnotesize 278.04} \textbf{0.44} & {\footnotesize 156.02} \textbf{0.79} & {\footnotesize 342.44} \textbf{0.36} & {\footnotesize 156.02} \textbf{0.79} & {\footnotesize 642} \textbf{0.19} & uns. & \multicolumn{1}{|r}{123}\\
\multicolumn{1}{l|}{p24} & {\footnotesize 262.04} \textbf{0.53} & {\footnotesize 184.02} \textbf{0.76} & uns. & {\footnotesize 184.02} \textbf{0.76} & {\footnotesize 1116} \textbf{0.13} & uns. & \multicolumn{1}{|r}{140}\\
\multicolumn{1}{l|}{p25} & {\footnotesize 373.05} \textbf{0.42} & {\footnotesize 199.02} \textbf{0.78} & uns. & {\footnotesize 191.02} \textbf{0.82} & uns. & uns. & \multicolumn{1}{|r}{156}\\
\multicolumn{1}{l|}{p26} & uns. & {\footnotesize 234.03} \textbf{1.00} & uns. & {\footnotesize 234.02} \textbf{1.00} & uns. & uns. & \multicolumn{1}{|r}{234.02}\\
\multicolumn{1}{l|}{p27} & uns. & {\footnotesize 254.03} \textbf{1.00} & uns. & {\footnotesize 256.03} \textbf{0.99} & uns. & uns. & \multicolumn{1}{|r}{254.03}\\
\multicolumn{1}{l|}{p28} & uns. & {\footnotesize 312.03} \textbf{1.00} & uns. & {\footnotesize 314.03} \textbf{0.99} & uns. & uns. & \multicolumn{1}{|r}{312.03}\\
\multicolumn{1}{l|}{p29} & uns. & {\footnotesize 314.03} \textbf{1.00} & uns. & {\footnotesize 314.03} \textbf{1.00} & uns. & uns. & \multicolumn{1}{|r}{314.03}\\
\multicolumn{1}{l|}{p30} & uns. & {\footnotesize 385.04} \textbf{0.90} & uns. & {\footnotesize 346.03} \textbf{1.00} & uns. & uns. & \multicolumn{1}{|r}{346.03}\\
\midrule
\textbf{total} & \textbf{14.50} & \textbf{27.16} & \textbf{11.05} & \textbf{28.02} & \textbf{7.35} & \textbf{3.43} & \\
\bottomrule
\end{tabular}


\caption{Quality and score of temporal planners on the tempo-sat-6 dataset.}
\label{tab:tempo-sat-6-ipc-scores}
\end{subtable}

\vspace{0.5cm}
\begin{subfigure}{\textwidth}
\centering
\includegraphics[width=1.0\textwidth]{../imga/tempo-sat-6-quality}
\caption{Quality plot of temporal planners on the tempo-sat-6 dataset.}
\label{fig:tempo-sat-6-quality}
\end{subfigure}
\caption{Planner results on tempo-sat-6.}
\label{fig:tempo-sat-6-results}
\end{figure}

\begin{figure}[tbp]
\centering
\begin{subfigure}{\textwidth}
\centering
\includegraphics[width=1.0\textwidth]{../imga/tempo-sat-6-gantt-p12-RRAPN}
\caption{Gantt chart of the temporal planner RRAPNSched on the tempo-sat-6 \texttt{p12} problem.}
\label{fig:tempo-sat-6-gantt-12-rrapn}
\end{subfigure}

\comment{
\begin{subfigure}{\textwidth}
\centering
\includegraphics[width=1.0\textwidth]{../imga/tempo-sat-6-gantt-p12-MSFA5}
\caption{Gantt chart of the temporal planner MSFA5Sched on the tempo-sat-6 \texttt{p12} problem.}
\label{fig:tempo-sat-6-gantt-12-msfa5}
\end{subfigure}
}

\begin{subfigure}{\textwidth}
\centering
\includegraphics[width=1.0\textwidth]{../imga/tempo-sat-6-gantt-p12-TFD}
\caption{Gantt chart of the temporal planner TFD2014 on the tempo-sat-6 \texttt{p12} problem.}
\label{fig:tempo-sat-6-gantt-12-tfd}
\end{subfigure}
\caption{Gantt charts of the temporal planner RRAPNSched and TFD2014 on the tempo-sat-6 \texttt{p12} problem.}
\label{fig:tempo-sat-6-gantt}
\end{figure}

Planners that entered the 2008 temporal track at the IPC did not cope well with the Transport domain
--- only two non-baseline planners (SGPlan$_6$ and TFD) were able to produce at least one plan
for any problem. Additionally, only the smallest problem, problem number 1, was solved
to the best known score by any planner.
The best total quality was only $7.5/30$, achieved by
{SGPlan$_6$. No other domain in the temporal track had a lower best total quality
than Transport, which, assuming reasonably generated problem instances, hints
at Transport being one of the harder domains for domain-independent temporal planners.
We observe an evident performance increase of Temporal Fast Downward,
when comparing the qualities of plans of TFD (from 2008) and TFD2014.

Our results further show that using a simple domain-dependent scheduling approach
of sequential plans yields an improvement over domain-independent temporal planners. Our planners RRAPNSched
and MSFA5Sched achieve total qualities of $25.54/30$
and $10.59/30$, respectively.
The scheduled MSFA5 planner does not generate plans of such variety as RRAPN,
and therefore produces worse results when scheduled using our algorithm,
mostly due to the inability of the scheduler to add enough \refuel{}
actions to the existing plans.

RRAPNSched, on the other hand, is able to beat even newer temporal planners like TFD2014
by a significant margin.
We see that RRAPNSched produces plans with worse scores than the best known score on smaller problems, which is mainly due to the fact that plans for smaller problems are easier to precalculate and hence the best known score estimate is closer
to the optimum than the estimates for larger problems.
An interesting case is problem number 10,
which was not solved by any planner, except TFD2014. We were not able to reasonably explain
this phenomenon. \TODO{explain + p21 explain or fix}

In Figure~\ref{fig:tempo-sat-6-gantt}, we see a comparison of the plans
of RRAPNSched and TFD2014 for problem number 12.
Observe that the important difference between the two plans
is that \texttt{truck-1} chooses to pick up and move \texttt{package-4}
while it is delivering \texttt{package-2}. It then later picks it up again and delivers
it while \texttt{truck-2} is delivering \texttt{package-3}.
In the plan of TFD2014, \texttt{truck-1} did not pick up \texttt{package-4}
while delivering \texttt{package-2} and then had to travel further to deliver it
--- this is basically the only difference between the two plans,
and it makes a difference in makespans of more than 50.
Finally, observe that due to the nature of RRAPN, \texttt{truck-1}
dropped \texttt{package-4} and then went back to pick it up, even though it had enough capacity
to carry it. Another important observation, however, is that this had no effect on the total makespan
of the plan.

\section{Overall results}

The attained results show that domain-specific information can be leveraged
to generate plans of better quality.
We have designed and implemented Transport planners that are able to beat
results from the original competitions in the sequential
and temporal satisficing tracks of the 2008 and 2011 IPCs.
In the 2014 IPC, we would have attained second place on overall quality in the Transport domain, behind the impressive result of Mercury.

Another major advantage previously unmentioned is that
our planners generate good solutions quite fast.
In Table~\ref{tab:planner-results-short},
we show results from a 10 second run of our planners on all datasets.


\begin{table}[tbp]
\centering
\begin{subtable}{0.48\textwidth}
\centering
\tiny
\renewcommand{\footnotesize}{\tiny}
\begin{tabular}{|l|rrr|r|}
\hline
\textbf{\#} & \textbf{MSFA3} & \textbf{MSFA5} & \textbf{RRAPN} & \textbf{BEST}\\
\hline
p01 & {\footnotesize 54} \textbf{1.00} & {\footnotesize 54} \textbf{1.00} & {\footnotesize 54} \textbf{1.00} & 54\\
p02 & {\footnotesize 270} \textbf{1.00} & {\footnotesize 270} \textbf{1.00} & {\footnotesize 288} \textbf{0.94} & 270\\
p03 & {\footnotesize 419} \textbf{0.88} & {\footnotesize 408} \textbf{0.90} & {\footnotesize 419} \textbf{0.88} & 369\\
p04 & {\footnotesize 464} \textbf{0.78} & {\footnotesize 490} \textbf{0.74} & {\footnotesize 412} \textbf{0.88} & 363\\
p05 & {\footnotesize 732} \textbf{0.82} & {\footnotesize 732} \textbf{0.82} & {\footnotesize 609} \textbf{0.98} & 597\\
p06 & {\footnotesize 989} \textbf{0.76} & {\footnotesize 967} \textbf{0.78} & {\footnotesize 1091} \textbf{0.69} & 755\\
p07 & {\footnotesize 1011} \textbf{1.00} & {\footnotesize 1011} \textbf{1.00} & {\footnotesize 1060} \textbf{0.95} & 1011\\
p08 & {\footnotesize 1053} \textbf{0.95} & {\footnotesize 1053} \textbf{0.95} & {\footnotesize 1004} \textbf{1.00} & 1004\\
p09 & {\footnotesize 1027} \textbf{0.97} & {\footnotesize 1027} \textbf{0.97} & {\footnotesize 1001} \textbf{1.00} & 1001\\
p10 & {\footnotesize 1360} \textbf{0.86} & {\footnotesize 1360} \textbf{0.86} & {\footnotesize 1164} \textbf{1.00} & 1164\\
p11 & {\footnotesize 473} \textbf{1.00} & {\footnotesize 473} \textbf{1.00} & {\footnotesize 473} \textbf{1.00} & 473\\
p12 & {\footnotesize 823} \textbf{0.97} & {\footnotesize 823} \textbf{0.97} & {\footnotesize 872} \textbf{0.91} & 795\\
p13 & {\footnotesize 1096} \textbf{0.88} & {\footnotesize 1096} \textbf{0.88} & {\footnotesize 965} \textbf{1.00} & 965\\
p14 & {\footnotesize 1582} \textbf{1.00} & {\footnotesize 1582} \textbf{1.00} & {\footnotesize 1966} \textbf{0.80} & 1582\\
p15 & {\footnotesize 2367} \textbf{0.96} & {\footnotesize 2280} \textbf{1.00} & {\footnotesize 3147} \textbf{0.72} & 2280\\
p16 & {\footnotesize 2321} \textbf{1.00} & {\footnotesize 2321} \textbf{1.00} & {\footnotesize 2899} \textbf{0.80} & 2321\\
p17 & {\footnotesize 3209} \textbf{1.00} & {\footnotesize 3209} \textbf{1.00} & {\footnotesize 4421} \textbf{0.73} & 3209\\
p18 & {\footnotesize 3406} \textbf{0.86} & {\footnotesize 2936} \textbf{1.00} & {\footnotesize 3754} \textbf{0.78} & 2936\\
p19 & {\footnotesize 5051} \textbf{1.00} & {\footnotesize 5051} \textbf{1.00} & {\footnotesize 5187} \textbf{0.97} & 5051\\
p20 & {\footnotesize 4189} \textbf{1.00} & {\footnotesize 4189} \textbf{1.00} & {\footnotesize 5234} \textbf{0.80} & 4189\\
p21 & {\footnotesize 431} \textbf{1.00} & {\footnotesize 431} \textbf{1.00} & {\footnotesize 431} \textbf{1.00} & 431\\
p22 & {\footnotesize 675} \textbf{1.00} & {\footnotesize 710} \textbf{0.95} & {\footnotesize 677} \textbf{1.00} & 675\\
p23 & {\footnotesize 1140} \textbf{0.73} & {\footnotesize 1140} \textbf{0.73} & {\footnotesize 897} \textbf{0.93} & 837\\
p24 & {\footnotesize 1227} \textbf{1.00} & {\footnotesize 1227} \textbf{1.00} & {\footnotesize 1423} \textbf{0.86} & 1227\\
p25 & {\footnotesize 1943} \textbf{0.94} & {\footnotesize 1943} \textbf{0.94} & {\footnotesize 1943} \textbf{0.94} & 1833\\
p26 & {\footnotesize 2421} \textbf{0.77} & {\footnotesize 2421} \textbf{0.77} & {\footnotesize 1871} \textbf{1.00} & 1871\\
p27 & {\footnotesize 3255} \textbf{0.81} & {\footnotesize 3255} \textbf{0.81} & {\footnotesize 2634} \textbf{1.00} & 2634\\
p28 & {\footnotesize 2465} \textbf{1.00} & {\footnotesize 2465} \textbf{1.00} & {\footnotesize 2807} \textbf{0.88} & 2465\\
p29 & {\footnotesize 2817} \textbf{1.00} & {\footnotesize 2890} \textbf{0.97} & {\footnotesize 3267} \textbf{0.86} & 2817\\
p30 & {\footnotesize 4703} \textbf{0.92} & {\footnotesize 4703} \textbf{0.92} & {\footnotesize 4330} \textbf{1.00} & 4330\\
\hline
\textbf{total} & \textbf{27.88} & \textbf{27.98} & \textbf{27.33} & \\
\hline
\end{tabular}


\caption{Quality and score of our planners on the seq-sat-6 dataset.}
\label{tab:seq-sat-6-ipc-scores-short}
\end{subtable}
\quad
\begin{subtable}{0.48\textwidth}
\centering
\tiny
\renewcommand{\footnotesize}{\tiny}
\begin{tabular}{|l|rrr|r|}
\hline
\textbf{\#} & \textbf{MSFA5Sched} & \textbf{RRAPNSched} & \textbf{TFD2014} & \textbf{BEST}\\
\hline
p01 & {\footnotesize 52} \textbf{1.00} & {\footnotesize 52} \textbf{1.00} & {\footnotesize 52.02} \textbf{1.00} & 52\\
p02 & {\footnotesize 150.01} \textbf{0.82} & {\footnotesize 126.01} \textbf{0.98} & {\footnotesize 150.11} \textbf{0.82} & 123\\
p03 & uns. & {\footnotesize 198.01} \textbf{0.95} & {\footnotesize 252.14} \textbf{0.75} & 189\\
p04 & {\footnotesize 341.02} \textbf{0.78} & {\footnotesize 267.01} \textbf{1.00} & {\footnotesize 425.29} \textbf{0.63} & 267.01\\
p05 & {\footnotesize 285.03} \textbf{0.87} & {\footnotesize 249.02} \textbf{1.00} & {\footnotesize 406.25} \textbf{0.61} & 249.02\\
p06 & {\footnotesize 316.02} \textbf{0.82} & {\footnotesize 284.02} \textbf{0.92} & uns. & 260\\
p07 & uns. & uns. & uns. & --\\
p08 & uns. & uns. & uns. & --\\
p09 & uns. & {\footnotesize 381.04} \textbf{1.00} & uns. & 381.04\\
p10 & uns. & uns. & uns. & --\\
p11 & {\footnotesize 332.01} \textbf{1.00} & {\footnotesize 332.01} \textbf{1.00} & {\footnotesize 342.09} \textbf{0.97} & 332.01\\
p12 & {\footnotesize 483.01} \textbf{1.00} & {\footnotesize 490.01} \textbf{0.99} & {\footnotesize 543.13} \textbf{0.89} & 483.01\\
p13 & {\footnotesize 572.02} \textbf{0.85} & {\footnotesize 484.01} \textbf{1.00} & {\footnotesize 1172.38} \textbf{0.41} & 484.01\\
p14 & {\footnotesize 777.03} \textbf{0.86} & {\footnotesize 669.02} \textbf{1.00} & {\footnotesize 1938.75} \textbf{0.35} & 669.02\\
p15 & {\footnotesize 1081.04} \textbf{0.86} & {\footnotesize 931.04} \textbf{1.00} & uns. & 931.04\\
p16 & {\footnotesize 1532.07} \textbf{0.69} & {\footnotesize 1059.05} \textbf{1.00} & uns. & 1059.05\\
p17 & {\footnotesize 1495.07} \textbf{0.78} & {\footnotesize 1170.06} \textbf{1.00} & uns. & 1170.06\\
p18 & uns. & {\footnotesize 1479.07} \textbf{1.00} & uns. & 1479.07\\
p19 & uns. & {\footnotesize 1733.1} \textbf{1.00} & uns. & 1733.1\\
p20 & uns. & {\footnotesize 2094.12} \textbf{1.00} & uns. & 2094.12\\
p21 & uns. & uns. & uns. & 63\\
p22 & uns. & {\footnotesize 208.03} \textbf{0.45} & {\footnotesize 265.38} \textbf{0.35} & 94\\
p23 & uns. & {\footnotesize 275.03} \textbf{0.45} & {\footnotesize 342.44} \textbf{0.36} & 123\\
p24 & {\footnotesize 350.04} \textbf{0.40} & {\footnotesize 202.02} \textbf{0.69} & uns. & 140\\
p25 & uns. & {\footnotesize 255.03} \textbf{0.61} & uns. & 156\\
p26 & uns. & {\footnotesize 335.04} \textbf{1.00} & uns. & 335.04\\
p27 & uns. & {\footnotesize 462.05} \textbf{1.00} & uns. & 462.05\\
p28 & uns. & {\footnotesize 619.06} \textbf{1.00} & uns. & 619.06\\
p29 & uns. & {\footnotesize 412.05} \textbf{1.00} & uns. & 412.05\\
p30 & uns. & uns. & uns. & --\\
\hline
\textbf{total} & \textbf{10.74} & \textbf{23.04} & \textbf{7.14} & \\
\hline
\end{tabular}


\caption{Quality and score of our planners on the tempo-sat-6 dataset. \TODO{remove TFD2014}}
\label{tab:tempo-sat-6-ipc-scores-short}
\end{subtable}

\vspace{0.5cm}
\begin{subtable}{0.48\textwidth}
\centering
\tiny
\renewcommand{\footnotesize}{\tiny}
\begin{tabular}{|l|rrr|r|}
\hline
\textbf{\#} & \textbf{MSFA3} & \textbf{MSFA5} & \textbf{RRAPN} & \textbf{BEST}\\
\hline
p01 & {\footnotesize 1053} \textbf{0.92} & {\footnotesize 1053} \textbf{0.92} & {\footnotesize 974} \textbf{1.00} & 974\\
p02 & {\footnotesize 1027} \textbf{0.94} & {\footnotesize 1027} \textbf{0.94} & {\footnotesize 963} \textbf{1.00} & 963\\
p03 & {\footnotesize 2817} \textbf{1.00} & {\footnotesize 2890} \textbf{0.97} & {\footnotesize 2997} \textbf{0.94} & 2817\\
p04 & {\footnotesize 2321} \textbf{1.00} & {\footnotesize 2321} \textbf{1.00} & {\footnotesize 2832} \textbf{0.82} & 2321\\
p05 & {\footnotesize 3209} \textbf{1.00} & {\footnotesize 3209} \textbf{1.00} & {\footnotesize 4386} \textbf{0.73} & 3209\\
p06 & {\footnotesize 3406} \textbf{0.86} & {\footnotesize 2936} \textbf{1.00} & {\footnotesize 3754} \textbf{0.78} & 2936\\
p07 & {\footnotesize 5051} \textbf{1.00} & {\footnotesize 5051} \textbf{1.00} & {\footnotesize 5187} \textbf{0.97} & 5051\\
p08 & {\footnotesize 1360} \textbf{0.84} & {\footnotesize 1360} \textbf{0.84} & {\footnotesize 1139} \textbf{1.00} & 1139\\
p09 & {\footnotesize 3636} \textbf{1.00} & {\footnotesize 3873} \textbf{0.94} & {\footnotesize 4924} \textbf{0.74} & 3636\\
p10 & {\footnotesize 4703} \textbf{0.84} & {\footnotesize 4703} \textbf{0.84} & {\footnotesize 3940} \textbf{1.00} & 3940\\
p11 & {\footnotesize 1426} \textbf{0.89} & {\footnotesize 1426} \textbf{0.89} & {\footnotesize 1271} \textbf{1.00} & 1271\\
p12 & {\footnotesize 1466} \textbf{1.00} & {\footnotesize 1466} \textbf{1.00} & {\footnotesize 1473} \textbf{1.00} & 1466\\
p13 & {\footnotesize 1630} \textbf{1.00} & {\footnotesize 1630} \textbf{1.00} & {\footnotesize 1853} \textbf{0.88} & 1630\\
p14 & {\footnotesize 5930} \textbf{1.00} & {\footnotesize 5930} \textbf{1.00} & {\footnotesize 6655} \textbf{0.89} & 5930\\
p15 & {\footnotesize 4984} \textbf{1.00} & {\footnotesize 4984} \textbf{1.00} & {\footnotesize 6175} \textbf{0.81} & 4984\\
p16 & inv. & inv. & {\footnotesize 6132} \textbf{1.00} & 6124\\
p17 & {\footnotesize 4838} \textbf{1.00} & {\footnotesize 4838} \textbf{1.00} & {\footnotesize 5126} \textbf{0.94} & 4838\\
p18 & {\footnotesize 3963} \textbf{0.98} & {\footnotesize 3963} \textbf{0.98} & {\footnotesize 3872} \textbf{1.00} & 3872\\
p19 & {\footnotesize 4124} \textbf{1.00} & {\footnotesize 4124} \textbf{1.00} & {\footnotesize 4474} \textbf{0.92} & 4124\\
p20 & {\footnotesize 3765} \textbf{1.00} & {\footnotesize 3765} \textbf{1.00} & {\footnotesize 4550} \textbf{0.83} & 3765\\
\hline
\textbf{total} & \textbf{19.27} & \textbf{19.32} & \textbf{18.25} & \\
\hline
\end{tabular}


\caption{Quality and score of our planners on the seq-sat-7 dataset.}
\label{tab:seq-sat-7-ipc-scores-short}
\end{subtable}
\quad
\begin{subtable}{0.48\textwidth}
\centering
\tiny
\renewcommand{\footnotesize}{\tiny}
\begin{tabular}{|l|rrr|r|}
\hline
\textbf{\#} & \textbf{MSFA3} & \textbf{MSFA5} & \textbf{RRAPN} & \textbf{BEST}\\
\hline
p01 & {\footnotesize 1596} \textbf{0.82} & {\footnotesize 1596} \textbf{0.82} & {\footnotesize 1825} \textbf{0.72} & 1309\\
p02 & {\footnotesize 2109} \textbf{1.00} & {\footnotesize 2109} \textbf{1.00} & {\footnotesize 2543} \textbf{0.83} & 2109\\
p03 & {\footnotesize 1879} \textbf{0.82} & {\footnotesize 1879} \textbf{0.82} & {\footnotesize 1898} \textbf{0.81} & 1539\\
p04 & uns. & uns. & {\footnotesize 7978} \textbf{0.71} & 5678\\
p05 & uns. & uns. & {\footnotesize 7730} \textbf{0.81} & 6235\\
p06 & uns. & uns. & {\footnotesize 7978} \textbf{0.71} & 5678\\
p07 & uns. & uns. & {\footnotesize 5807} \textbf{0.83} & 4839\\
p08 & uns. & {\footnotesize 4996} \textbf{0.89} & {\footnotesize 5848} \textbf{0.76} & 4467\\
p09 & uns. & uns. & {\footnotesize 5807} \textbf{0.83} & 4839\\
p10 & {\footnotesize 4473} \textbf{1.00} & {\footnotesize 4473} \textbf{1.00} & {\footnotesize 6175} \textbf{0.72} & 4473\\
p11 & {\footnotesize 1395} \textbf{0.96} & {\footnotesize 1395} \textbf{0.96} & {\footnotesize 1677} \textbf{0.80} & 1336\\
p12 & {\footnotesize 1579} \textbf{1.00} & {\footnotesize 1579} \textbf{1.00} & {\footnotesize 2263} \textbf{0.70} & 1579\\
p13 & {\footnotesize 1683} \textbf{0.68} & {\footnotesize 1683} \textbf{0.68} & {\footnotesize 1673} \textbf{0.69} & 1147\\
p14 & uns. & uns. & {\footnotesize 7708} \textbf{0.78} & 5974\\
p15 & uns. & uns. & {\footnotesize 8136} \textbf{0.65} & 5320\\
p16 & {\footnotesize 5179} \textbf{0.91} & {\footnotesize 5179} \textbf{0.91} & {\footnotesize 7038} \textbf{0.67} & 4695\\
p17 & uns. & uns. & {\footnotesize 5739} \textbf{0.79} & 4540\\
p18 & {\footnotesize 4585} \textbf{1.00} & {\footnotesize 4585} \textbf{1.00} & {\footnotesize 6160} \textbf{0.74} & 4585\\
p19 & {\footnotesize 3812} \textbf{1.00} & {\footnotesize 3812} \textbf{1.00} & {\footnotesize 4837} \textbf{0.79} & 3812\\
p20 & {\footnotesize 4173} \textbf{0.92} & {\footnotesize 4173} \textbf{0.92} & {\footnotesize 5312} \textbf{0.73} & 3853\\
\hline
\textbf{total} & \textbf{10.11} & \textbf{11.00} & \textbf{15.07} & \\
\hline
\end{tabular}


\caption{Quality and score of our planners on the seq-sat-8 dataset.}
\label{tab:seq-sat-8-ipc-scores-short}
\end{subtable}
\caption{Results of our planners when run for 10 seconds on all datasets.}
\label{tab:planner-results-short}
\end{table}

