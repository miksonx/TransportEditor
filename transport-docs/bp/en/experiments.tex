\chapter{Experimental evaluation}

In this chapter, we will describe and run experiments
that compare our planners from the last two chapters
with state-of-the-art domain-independent planners from the IPC.
We will briefly discuss the acquired results and interpret them.

\section{Methodology}

Using our benchmarking software (described in the attachments), we will now run experiments in an
environment as similar to the original
IPC as possible, following almost all of their rules mentioned in.\footnote{\url{http://icaps-conference.org/ipc2008/deterministic/CompetitionRules.html}}
All planners are be single-threaded and use a maximum of 2GB of memory, with a maximum run time of 30 minutes (planners will
get canceled and prompted for a plan at that time point).
Our planners explicitly and intentionally break the ``domain-independence'' rule of the IPC.

The evaluation criteria remain the same:
we focus on plan \textit{quality} in favor of planner run time,
although we will mention runtimes.
The quality of a plan for a specific planner and sequential problem $p$ is defined as
$$\frac{\mt{total-cost}(\mt{planner}(p))}{\mt{total-cost}(BEST)},$$
where the results called $BEST$
are either precalculated outside of the competition environment or they are the best result of one of the planners in the competition, depending on which plan has a lower total cost.
Quality is, therefore, a number between $0$ and $1$.
Note that we will use the original competition's best scores,
which means that the quality of our plans
might sometimes exceed the value of 1
(when our planners find a plan better than all the others found during the competition).
The overall goal for planners is to maximize the sum of qualities over the problem instances in a given dataset, called the \textit{total quality}.
For temporal domains, quality is calculated in the same way, just by substituting total cost
for total time. We sometimes refer to total cost and total time as the \textit{score} of the planner, a term that is not dependent on the domain variant.

We will use four datasets for our experiments --- the seq-sat-6, seq-sat-7,
and seq-sat-8 datasets for sequential
and the tempo-sat-6 for temporal planners (Section~\ref{datasets}).
All the datasets used (and more) are available in the
software project sources (see the attached \nameref{cd-contents}).
Descriptions of planners that we will refer to by their competition names can be found in the respective competition results or booklets for IPC~2008\footnote{\url{http://icaps-conference.org/ipc2008/deterministic/Planners.html}}, IPC~2011 \citep{Garcia-Olaya2011} and IPC~2014 \citep{Vallati2015}.

In all planners where nondeterminism occurs,
we set the initial random seed to 2017
(on all individual problem runs).
All the following experiments
were run on a computer
containing an 8 core 64-bit processor \texttt{Intel(R) Core(TM) i7-6700 CPU @ 3.40GHz}
with 16 GB of memory, running Gentoo Linux.
A big thank you goes out to the faculty for providing us with access to these machines.
We run all Java programs on Oracle's OpenJDK 
version \texttt{1.8.0\_121}, build \texttt{b13}.
The results presented here were obtained with the \TEver{} version of the TransportEditor project.\footnote{Git tag \TEtag{}, available at \url{https://github.com/oskopek/TransportEditor}} The \texttt{NOTICE.txt} files
in the project module directories specify
the exact versions of libraries used.



















\section{Sequential Transport}

In this section, we present the results of our sequential planners on the seq-sat-6 and seq-sat-8 datasets. Specifically, these planners are included in the experiment:
\begin{description}
\item[SFA3] The SFA* algorithm with heuristic C (Section~\ref{sfa3})
\item[MSFA3] The metaheuristically weighted SFA3 planner (Section~\ref{msfa})
\item[RRAPN] The Randomized Restart Around Path Nearby Planner (Section~\ref{rrapn})
\item[CSP] The \TODO{CSP planner} (Section~\ref{csp-planner})
\end{description}

\subsection{Results}\label{sequential-results}

We show an IPC quality table and a quality plot
for the experimental runs on the seq-sat-6 dataset (Figure~\ref{fig:seq-sat-6-results}), seq-sat-7 dataset (Figure~\ref{fig:seq-sat-7-results}), and the seq-sat-8 dataset (Figure~\ref{fig:seq-sat-8-results}). Details about the specific plans along with the benchmark results can be found in the \nameref{cd-contents}.

\begin{figure}[p]
\centering
\begin{subtable}{\textwidth}
\centering
\scriptsize
\renewcommand{\footnotesize}{\scriptsize}
\begin{tabular}{|l|rrrrrr|r|}
\hline
\textbf{\#} & \textbf{MSFA3} & \textbf{MSFA5} & \textbf{RRAPN} & \textbf{dtg-plan} & \textbf{lama} & \textbf{sgplan6} & \textbf{BEST}\\
\hline
p01 & {\footnotesize 54} \textbf{1.00} & {\footnotesize 54} \textbf{1.00} & {\footnotesize 54} \textbf{1.00} & {\footnotesize 54} \textbf{1.00} & {\footnotesize 54} \textbf{1.00} & {\footnotesize 54} \textbf{1.00} & 54\\
p02 & {\footnotesize 270} \textbf{1.00} & {\footnotesize 270} \textbf{1.00} & {\footnotesize 288} \textbf{0.94} & {\footnotesize 304} \textbf{0.89} & {\footnotesize 270} \textbf{1.00} & {\footnotesize 414} \textbf{0.65} & 270\\
p03 & {\footnotesize 409} \textbf{0.87} & {\footnotesize 355} \textbf{1.00} & {\footnotesize 419} \textbf{0.85} & {\footnotesize 635} \textbf{0.56} & {\footnotesize 497} \textbf{0.71} & {\footnotesize 801} \textbf{0.44} & 355\\
p04 & {\footnotesize 464} \textbf{0.78} & {\footnotesize 490} \textbf{0.74} & {\footnotesize 412} \textbf{0.88} & {\footnotesize 983} \textbf{0.37} & {\footnotesize 504} \textbf{0.72} & {\footnotesize 942} \textbf{0.39} & 363\\
p05 & {\footnotesize 704} \textbf{0.83} & {\footnotesize 704} \textbf{0.83} & {\footnotesize 582} \textbf{1.00} & {\footnotesize 1187} \textbf{0.49} & {\footnotesize 737} \textbf{0.79} & {\footnotesize 1186} \textbf{0.49} & 582\\
p06 & {\footnotesize 989} \textbf{0.76} & {\footnotesize 967} \textbf{0.78} & {\footnotesize 1026} \textbf{0.74} & {\footnotesize 1766} \textbf{0.43} & {\footnotesize 1117} \textbf{0.68} & {\footnotesize 1868} \textbf{0.40} & 755\\
p07 & {\footnotesize 1011} \textbf{0.98} & {\footnotesize 1011} \textbf{0.98} & {\footnotesize 988} \textbf{1.00} & {\footnotesize 1868} \textbf{0.53} & {\footnotesize 1260} \textbf{0.78} & {\footnotesize 2081} \textbf{0.47} & 988\\
p08 & {\footnotesize 1053} \textbf{0.88} & {\footnotesize 1053} \textbf{0.88} & {\footnotesize 925} \textbf{1.00} & {\footnotesize 2166} \textbf{0.43} & {\footnotesize 1216} \textbf{0.76} & {\footnotesize 2135} \textbf{0.43} & 925\\
p09 & {\footnotesize 1027} \textbf{0.93} & {\footnotesize 1027} \textbf{0.93} & {\footnotesize 955} \textbf{1.00} & {\footnotesize 1880} \textbf{0.51} & {\footnotesize 1001} \textbf{0.95} & {\footnotesize 2143} \textbf{0.45} & 955\\
p10 & {\footnotesize 1360} \textbf{0.78} & {\footnotesize 1360} \textbf{0.78} & {\footnotesize 1059} \textbf{1.00} & {\footnotesize 2260} \textbf{0.47} & {\footnotesize 1285} \textbf{0.82} & {\footnotesize 2091} \textbf{0.51} & 1059\\
p11 & {\footnotesize 473} \textbf{1.00} & {\footnotesize 473} \textbf{1.00} & {\footnotesize 473} \textbf{1.00} & {\footnotesize 473} \textbf{1.00} & {\footnotesize 473} \textbf{1.00} & {\footnotesize 475} \textbf{1.00} & 473\\
p12 & {\footnotesize 823} \textbf{0.97} & {\footnotesize 823} \textbf{0.97} & {\footnotesize 872} \textbf{0.91} & {\footnotesize 800} \textbf{0.99} & {\footnotesize 795} \textbf{1.00} & {\footnotesize 1244} \textbf{0.64} & 795\\
p13 & {\footnotesize 1096} \textbf{0.88} & {\footnotesize 1096} \textbf{0.88} & {\footnotesize 965} \textbf{1.00} & {\footnotesize 2751} \textbf{0.35} & {\footnotesize 1147} \textbf{0.84} & {\footnotesize 2827} \textbf{0.34} & 965\\
p14 & {\footnotesize 1582} \textbf{1.00} & {\footnotesize 1582} \textbf{1.00} & {\footnotesize 1966} \textbf{0.80} & {\footnotesize 3507} \textbf{0.45} & {\footnotesize 2157} \textbf{0.73} & {\footnotesize 3328} \textbf{0.48} & 1582\\
p15 & {\footnotesize 2367} \textbf{0.96} & {\footnotesize 2280} \textbf{1.00} & {\footnotesize 3129} \textbf{0.73} & {\footnotesize 5221} \textbf{0.44} & {\footnotesize 2954} \textbf{0.77} & {\footnotesize 5659} \textbf{0.40} & 2280\\
p16 & {\footnotesize 2321} \textbf{1.00} & {\footnotesize 2321} \textbf{1.00} & {\footnotesize 2765} \textbf{0.84} & {\footnotesize 6199} \textbf{0.37} & {\footnotesize 4928} \textbf{0.47} & {\footnotesize 6144} \textbf{0.38} & 2321\\
p17 & {\footnotesize 3209} \textbf{1.00} & {\footnotesize 3209} \textbf{1.00} & {\footnotesize 4321} \textbf{0.74} & {\footnotesize 7239} \textbf{0.44} & {\footnotesize 4193} \textbf{0.77} & {\footnotesize 7494} \textbf{0.43} & 3209\\
p18 & {\footnotesize 3322} \textbf{0.88} & {\footnotesize 2936} \textbf{1.00} & {\footnotesize 3663} \textbf{0.80} & {\footnotesize 7542} \textbf{0.39} & {\footnotesize 4151} \textbf{0.71} & {\footnotesize 7737} \textbf{0.38} & 2936\\
p19 & {\footnotesize 5051} \textbf{1.00} & {\footnotesize 5051} \textbf{1.00} & {\footnotesize 5073} \textbf{1.00} & {\footnotesize 9921} \textbf{0.51} & {\footnotesize 7648} \textbf{0.66} & {\footnotesize 8991} \textbf{0.56} & 5051\\
p20 & {\footnotesize 3636} \textbf{1.00} & {\footnotesize 3873} \textbf{0.94} & {\footnotesize 4674} \textbf{0.78} & uns. & {\footnotesize 6773} \textbf{0.54} & {\footnotesize 8663} \textbf{0.42} & 3636\\
p21 & {\footnotesize 431} \textbf{1.00} & {\footnotesize 431} \textbf{1.00} & {\footnotesize 431} \textbf{1.00} & {\footnotesize 431} \textbf{1.00} & {\footnotesize 431} \textbf{1.00} & {\footnotesize 431} \textbf{1.00} & 431\\
p22 & {\footnotesize 675} \textbf{1.00} & {\footnotesize 675} \textbf{1.00} & {\footnotesize 677} \textbf{1.00} & {\footnotesize 679} \textbf{0.99} & {\footnotesize 675} \textbf{1.00} & {\footnotesize 1268} \textbf{0.53} & 675\\
p23 & {\footnotesize 1140} \textbf{0.73} & {\footnotesize 1140} \textbf{0.73} & {\footnotesize 897} \textbf{0.93} & {\footnotesize 2414} \textbf{0.35} & {\footnotesize 837} \textbf{1.00} & {\footnotesize 2119} \textbf{0.39} & 837\\
p24 & {\footnotesize 1227} \textbf{1.00} & {\footnotesize 1227} \textbf{1.00} & {\footnotesize 1352} \textbf{0.91} & {\footnotesize 2790} \textbf{0.44} & {\footnotesize 1301} \textbf{0.94} & {\footnotesize 2909} \textbf{0.42} & 1227\\
p25 & {\footnotesize 1943} \textbf{0.92} & {\footnotesize 1943} \textbf{0.92} & {\footnotesize 1785} \textbf{1.00} & {\footnotesize 4007} \textbf{0.45} & {\footnotesize 1833} \textbf{0.97} & {\footnotesize 3764} \textbf{0.47} & 1785\\
p26 & {\footnotesize 2421} \textbf{0.74} & {\footnotesize 2421} \textbf{0.74} & {\footnotesize 1797} \textbf{1.00} & {\footnotesize 4036} \textbf{0.45} & {\footnotesize 2502} \textbf{0.72} & {\footnotesize 3598} \textbf{0.50} & 1797\\
p27 & {\footnotesize 3255} \textbf{0.77} & {\footnotesize 3255} \textbf{0.77} & {\footnotesize 2521} \textbf{1.00} & {\footnotesize 5791} \textbf{0.44} & {\footnotesize 3317} \textbf{0.76} & {\footnotesize 5948} \textbf{0.42} & 2521\\
p28 & {\footnotesize 2465} \textbf{1.00} & {\footnotesize 2465} \textbf{1.00} & {\footnotesize 2594} \textbf{0.95} & {\footnotesize 6346} \textbf{0.39} & {\footnotesize 3027} \textbf{0.81} & {\footnotesize 7300} \textbf{0.34} & 2465\\
p29 & {\footnotesize 2817} \textbf{1.00} & {\footnotesize 2890} \textbf{0.97} & {\footnotesize 2871} \textbf{0.98} & {\footnotesize 7168} \textbf{0.39} & {\footnotesize 3294} \textbf{0.86} & {\footnotesize 7237} \textbf{0.39} & 2817\\
p30 & {\footnotesize 4703} \textbf{0.76} & {\footnotesize 4703} \textbf{0.76} & {\footnotesize 3595} \textbf{1.00} & uns. & {\footnotesize 5513} \textbf{0.65} & {\footnotesize 7892} \textbf{0.46} & 3595\\
\hline
\textbf{total} & \textbf{27.43} & \textbf{27.61} & \textbf{27.77} & \textbf{15.50} & \textbf{24.43} & \textbf{15.19} & \\
\hline
\end{tabular}


\caption{Quality and score of sequential planners on the seq-sat-6 dataset.}
\label{tab:seq-sat-6-ipc-scores}
\end{subtable}

\vspace{0.5cm}
\begin{subfigure}{\textwidth}
\centering
\includegraphics[width=1.0\textwidth]{../imga/seq-sat-6-quality}
\caption{Quality plot of sequential planners on the seq-sat-6 dataset.}
\label{fig:seq-sat-6-quality}
\end{subfigure}
\caption{Planner results on seq-sat-6.}
\label{fig:seq-sat-6-results}
\end{figure}

\begin{figure}[p]
\centering
\begin{subtable}{\textwidth}
\centering
\scriptsize
\renewcommand{\footnotesize}{\scriptsize}
\begin{tabular}{|l|rrrrrrr|r|}
\hline
\textbf{Problem} & \textbf{MSFA3} & \textbf{RRAPN} & \textbf{SFA3} & \textbf{dae\_yahsp} & \textbf{lama-2008} & \textbf{lama-2011} & \textbf{roamer} & \textbf{BEST}\\
\hline
p01 & uns. & {\footnotesize 915} \textbf{1.00} & uns. & {\footnotesize 1498} \textbf{0.61} & {\footnotesize 1050} \textbf{0.87} & {\footnotesize 1485} \textbf{0.62} & {\footnotesize 1050} \textbf{0.87} & 915\\
p02 & {\footnotesize 992} \textbf{0.94} & {\footnotesize 936} \textbf{1.00} & uns. & {\footnotesize 1701} \textbf{0.55} & {\footnotesize 996} \textbf{0.94} & {\footnotesize 1010} \textbf{0.93} & {\footnotesize 996} \textbf{0.94} & 936\\
p03 & uns. & uns. & uns. & {\footnotesize 4971} \textbf{0.66} & {\footnotesize 3313} \textbf{0.99} & {\footnotesize 3882} \textbf{0.84} & {\footnotesize 3275} \textbf{1.00} & 3275\\
p04 & uns. & {\footnotesize 2768} \textbf{0.95} & uns. & {\footnotesize 2618} \textbf{1.00} & {\footnotesize 5135} \textbf{0.51} & {\footnotesize 3741} \textbf{0.70} & {\footnotesize 5841} \textbf{0.45} & 2618\\
p05 & uns. & {\footnotesize 4331} \textbf{0.85} & uns. & {\footnotesize 3661} \textbf{1.00} & {\footnotesize 5481} \textbf{0.67} & {\footnotesize 4805} \textbf{0.76} & {\footnotesize 5553} \textbf{0.66} & 3661\\
p06 & uns. & {\footnotesize 3703} \textbf{0.92} & uns. & {\footnotesize 3974} \textbf{0.86} & {\footnotesize 4320} \textbf{0.79} & {\footnotesize 5415} \textbf{0.63} & {\footnotesize 4681} \textbf{0.73} & 3401\\
p07 & uns. & {\footnotesize 5040} \textbf{1.00} & uns. & {\footnotesize 5683} \textbf{0.89} & {\footnotesize 6652} \textbf{0.76} & {\footnotesize 7222} \textbf{0.70} & {\footnotesize 7403} \textbf{0.68} & 5040\\
p08 & uns. & {\footnotesize 1051} \textbf{1.00} & uns. & {\footnotesize 2067} \textbf{0.51} & {\footnotesize 1211} \textbf{0.87} & {\footnotesize 1452} \textbf{0.72} & {\footnotesize 1211} \textbf{0.87} & 1051\\
p09 & uns. & {\footnotesize 4706} \textbf{1.00} & uns. & {\footnotesize 5863} \textbf{0.80} & {\footnotesize 6786} \textbf{0.69} & {\footnotesize 6479} \textbf{0.73} & {\footnotesize 6806} \textbf{0.69} & 4706\\
p10 & uns. & {\footnotesize 3686} \textbf{1.00} & uns. & {\footnotesize 6145} \textbf{0.60} & {\footnotesize 5943} \textbf{0.62} & {\footnotesize 5641} \textbf{0.65} & {\footnotesize 5445} \textbf{0.68} & 3686\\
p11 & uns. & {\footnotesize 1222} \textbf{1.00} & uns. & {\footnotesize 2139} \textbf{0.57} & {\footnotesize 1547} \textbf{0.79} & {\footnotesize 2113} \textbf{0.58} & {\footnotesize 1901} \textbf{0.64} & 1222\\
p12 & uns. & {\footnotesize 1408} \textbf{1.00} & uns. & {\footnotesize 2577} \textbf{0.55} & {\footnotesize 1929} \textbf{0.73} & {\footnotesize 1947} \textbf{0.72} & {\footnotesize 1915} \textbf{0.74} & 1408\\
p13 & uns. & {\footnotesize 1730} \textbf{1.00} & uns. & {\footnotesize 2860} \textbf{0.60} & uns. & {\footnotesize 2932} \textbf{0.59} & {\footnotesize 2746} \textbf{0.63} & 1730\\
p14 & uns. & {\footnotesize 6290} \textbf{1.00} & uns. & {\footnotesize 9004} \textbf{0.70} & {\footnotesize 7925} \textbf{0.79} & {\footnotesize 8493} \textbf{0.74} & {\footnotesize 7940} \textbf{0.79} & 6290\\
p15 & uns. & {\footnotesize 6070} \textbf{1.00} & uns. & {\footnotesize 7209} \textbf{0.84} & {\footnotesize 7192} \textbf{0.84} & {\footnotesize 6909} \textbf{0.88} & {\footnotesize 6924} \textbf{0.88} & 6070\\
p16 & uns. & {\footnotesize 5745} \textbf{1.00} & uns. & {\footnotesize 7533} \textbf{0.76} & {\footnotesize 6951} \textbf{0.83} & uns. & uns. & 5745\\
p17 & uns. & {\footnotesize 4929} \textbf{1.00} & uns. & {\footnotesize 9466} \textbf{0.52} & {\footnotesize 6166} \textbf{0.80} & {\footnotesize 5899} \textbf{0.84} & {\footnotesize 5209} \textbf{0.95} & 4929\\
p18 & uns. & {\footnotesize 3666} \textbf{1.00} & uns. & {\footnotesize 6659} \textbf{0.55} & {\footnotesize 5381} \textbf{0.68} & {\footnotesize 5690} \textbf{0.64} & {\footnotesize 3902} \textbf{0.94} & 3666\\
p19 & uns. & {\footnotesize 4277} \textbf{1.00} & uns. & {\footnotesize 8011} \textbf{0.53} & {\footnotesize 5716} \textbf{0.75} & {\footnotesize 5777} \textbf{0.74} & {\footnotesize 5257} \textbf{0.81} & 4277\\
p20 & uns. & {\footnotesize 4174} \textbf{1.00} & uns. & {\footnotesize 6891} \textbf{0.61} & {\footnotesize 5831} \textbf{0.72} & {\footnotesize 4435} \textbf{0.94} & {\footnotesize 4793} \textbf{0.87} & 4174\\
\hline
\textbf{total} & \textbf{0.94} & \textbf{18.71} & \textbf{0.00} & \textbf{13.71} & \textbf{14.63} & \textbf{13.95} & \textbf{14.81} & \\
\hline
\end{tabular}


\caption{Quality and score of sequential planners on the seq-sat-7 dataset.}
\label{tab:seq-sat-7-ipc-scores}
\end{subtable}

\vspace{0.5cm}
\begin{subfigure}{\textwidth}
\centering
\includegraphics[width=1.0\textwidth]{../imga/seq-sat-7-quality}
\caption{Quality plot of sequential planners on the seq-sat-7 dataset.}
\label{fig:seq-sat-7-quality}
\end{subfigure}
\caption{Planner results on seq-sat-7.}
\label{fig:seq-sat-7-results}
\end{figure}

\begin{figure}[p]
\centering
\begin{subtable}{\textwidth}
\centering
\scriptsize
\renewcommand{\footnotesize}{\scriptsize}
\begin{tabular}{|l|rrrrrr|r|}
\hline
\textbf{\#} & \textbf{MSFA3} & \textbf{MSFA5} & \textbf{RRAPN} & \textbf{ibacop} & \textbf{mercury} & \textbf{yahsp3-mt} & \textbf{BEST}\\
\hline
p01 & {\footnotesize 1596} \textbf{0.82} & {\footnotesize 1596} \textbf{0.82} & {\footnotesize 1600} \textbf{0.82} & {\footnotesize 2045} \textbf{0.64} & {\footnotesize 1309} \textbf{1.00} & {\footnotesize 3044} \textbf{0.43} & 1309\\
p02 & {\footnotesize 2109} \textbf{1.00} & {\footnotesize 2109} \textbf{1.00} & {\footnotesize 2300} \textbf{0.92} & {\footnotesize 5902} \textbf{0.36} & {\footnotesize 2125} \textbf{0.99} & {\footnotesize 4250} \textbf{0.50} & 2109\\
p03 & {\footnotesize 1879} \textbf{0.82} & {\footnotesize 1879} \textbf{0.82} & {\footnotesize 1785} \textbf{0.86} & {\footnotesize 2653} \textbf{0.58} & {\footnotesize 1539} \textbf{1.00} & {\footnotesize 3274} \textbf{0.47} & 1539\\
p04 & {\footnotesize 5163} \textbf{0.99} & {\footnotesize 5092} \textbf{1.00} & {\footnotesize 7508} \textbf{0.68} & {\footnotesize 8871} \textbf{0.57} & {\footnotesize 5678} \textbf{0.90} & {\footnotesize 8228} \textbf{0.62} & 5092\\
p05 & {\footnotesize 5394} \textbf{1.00} & {\footnotesize 5708} \textbf{0.94} & {\footnotesize 7518} \textbf{0.72} & {\footnotesize 14170} \textbf{0.38} & {\footnotesize 6235} \textbf{0.87} & {\footnotesize 10938} \textbf{0.49} & 5394\\
p06 & {\footnotesize 5163} \textbf{0.99} & {\footnotesize 5092} \textbf{1.00} & {\footnotesize 7508} \textbf{0.68} & {\footnotesize 8871} \textbf{0.57} & {\footnotesize 5678} \textbf{0.90} & {\footnotesize 8228} \textbf{0.62} & 5092\\
p07 & {\footnotesize 4202} \textbf{1.00} & {\footnotesize 4202} \textbf{1.00} & {\footnotesize 4996} \textbf{0.84} & {\footnotesize 11802} \textbf{0.36} & {\footnotesize 4839} \textbf{0.87} & {\footnotesize 7804} \textbf{0.54} & 4202\\
p08 & {\footnotesize 4996} \textbf{0.89} & {\footnotesize 4948} \textbf{0.90} & {\footnotesize 5276} \textbf{0.85} & {\footnotesize 12762} \textbf{0.35} & {\footnotesize 4467} \textbf{1.00} & {\footnotesize 8590} \textbf{0.52} & 4467\\
p09 & {\footnotesize 4202} \textbf{1.00} & {\footnotesize 4202} \textbf{1.00} & {\footnotesize 4996} \textbf{0.84} & {\footnotesize 11802} \textbf{0.36} & {\footnotesize 4839} \textbf{0.87} & {\footnotesize 7680} \textbf{0.55} & 4202\\
p10 & {\footnotesize 4473} \textbf{1.00} & {\footnotesize 4473} \textbf{1.00} & {\footnotesize 5731} \textbf{0.78} & {\footnotesize 8260} \textbf{0.54} & {\footnotesize 4626} \textbf{0.97} & {\footnotesize 8410} \textbf{0.53} & 4473\\
p11 & {\footnotesize 1395} \textbf{0.96} & {\footnotesize 1395} \textbf{0.96} & {\footnotesize 1555} \textbf{0.86} & {\footnotesize 2154} \textbf{0.62} & {\footnotesize 1336} \textbf{1.00} & {\footnotesize 2429} \textbf{0.55} & 1336\\
p12 & {\footnotesize 1579} \textbf{1.00} & {\footnotesize 1579} \textbf{1.00} & {\footnotesize 2073} \textbf{0.76} & {\footnotesize 2524} \textbf{0.63} & {\footnotesize 1641} \textbf{0.96} & {\footnotesize 3646} \textbf{0.43} & 1579\\
p13 & {\footnotesize 1683} \textbf{0.68} & {\footnotesize 1683} \textbf{0.68} & {\footnotesize 1537} \textbf{0.75} & {\footnotesize 2085} \textbf{0.55} & {\footnotesize 1147} \textbf{1.00} & {\footnotesize 3700} \textbf{0.31} & 1147\\
p14 & {\footnotesize 7196} \textbf{0.83} & {\footnotesize 7196} \textbf{0.83} & {\footnotesize 6764} \textbf{0.88} & {\footnotesize 10667} \textbf{0.56} & {\footnotesize 5974} \textbf{1.00} & {\footnotesize 9334} \textbf{0.64} & 5974\\
p15 & {\footnotesize 7671} \textbf{0.69} & {\footnotesize 7671} \textbf{0.69} & {\footnotesize 7906} \textbf{0.67} & {\footnotesize 12975} \textbf{0.41} & {\footnotesize 5320} \textbf{1.00} & {\footnotesize 11822} \textbf{0.45} & 5320\\
p16 & {\footnotesize 5179} \textbf{0.91} & {\footnotesize 5107} \textbf{0.92} & {\footnotesize 6914} \textbf{0.68} & {\footnotesize 10918} \textbf{0.43} & {\footnotesize 4695} \textbf{1.00} & {\footnotesize 8536} \textbf{0.55} & 4695\\
p17 & {\footnotesize 4823} \textbf{0.94} & {\footnotesize 4646} \textbf{0.98} & {\footnotesize 5371} \textbf{0.85} & {\footnotesize 9659} \textbf{0.47} & {\footnotesize 4540} \textbf{1.00} & {\footnotesize 8107} \textbf{0.56} & 4540\\
p18 & {\footnotesize 4585} \textbf{1.00} & {\footnotesize 4585} \textbf{1.00} & {\footnotesize 5681} \textbf{0.81} & {\footnotesize 10755} \textbf{0.43} & {\footnotesize 4840} \textbf{0.95} & {\footnotesize 10521} \textbf{0.44} & 4585\\
p19 & {\footnotesize 3812} \textbf{1.00} & {\footnotesize 3812} \textbf{1.00} & {\footnotesize 4837} \textbf{0.79} & {\footnotesize 10780} \textbf{0.35} & {\footnotesize 3881} \textbf{0.98} & {\footnotesize 7322} \textbf{0.52} & 3812\\
p20 & {\footnotesize 4173} \textbf{0.92} & {\footnotesize 3923} \textbf{0.98} & {\footnotesize 4991} \textbf{0.77} & {\footnotesize 9632} \textbf{0.40} & {\footnotesize 3853} \textbf{1.00} & {\footnotesize 6643} \textbf{0.58} & 3853\\
\hline
\textbf{total} & \textbf{18.44} & \textbf{18.53} & \textbf{15.80} & \textbf{9.56} & \textbf{19.25} & \textbf{10.29} & \\
\hline
\end{tabular}


\caption{Quality and score of sequential planners on the seq-sat-8 dataset.}
\label{tab:seq-sat-8-ipc-scores}
\end{subtable}

\vspace{0.5cm}
\begin{subfigure}{\textwidth}
\centering
\includegraphics[width=1.0\textwidth]{../imga/seq-sat-8-quality}
\caption{Quality plot of sequential planners on the seq-sat-8 dataset.}
\label{fig:seq-sat-8-quality}
\end{subfigure}
\caption{Planner results on seq-sat-8.}
\label{fig:seq-sat-8-results}
\end{figure}


\subsection{Discussion}

In the updated results of the sequential satisficing track of IPC 2008\footnote{\url{http://icaps-conference.org/ipc2008/deterministic/Results.html}} published after the competition,
the overall winner \textit{LAMA} (a Fast Downward based planner)
was hands-down the best planner on the sequential Transport domain, winning
with a total quality of $28.93/30$, where all other planners had less than $20/30$.
Only 5 problem instances were solved suboptimally by \textit{LAMA}.

As mentioned, we were not able to gather data from the IPC 2011,
hence the results cannot be meaningfully interpreted.
The competition featured 20 sequential Transport problems,
with 4 planners achieving a total quality of more than $15/20$.

In the satisficing track of IPC 2014, the winner on the Transport domain
was without a doubt the \textit{Mercury} planner, achieving
a stunning $20/20$ total quality. Even more interesting is the fact that
the runner-up \textit{yahsp3-mt} achieved a score of $10.74/20$
and all other planners achieved sub $10/20$ total quality.
The IPC 2014 used different problem instances than the IPC 2008
and we have tested our approaches on both datasets.

\TODO{try to analyze why} \TODO{Vytiahnut si plany z vysledkov seq-sat08 z LAMA na seqsat ipc08 an z mercury a lama na ipc14 + analysis of specific errors and shortcommings}


\TODO{our results show that...}













\section{Temporal Transport}

In this section, we present the results of our temporal planners on the tempo-sat-6 dataset. The following planners are included in the experiment:
\begin{description}
\item[RRAPNSched] The scheduled (Section~\ref{sched}) RRAPN planner (Section~\ref{rrapn})
\item[CSPT] The \TODO{temporal CSP planner} \TODO{ref}
\item[TFD2014] The Temporal Fast Downward planner, version 0.4 from IPC~2014 \citep[Preferring Preferred Operators in Temporal Fast Downward]{Vallati2015}
\end{description}

\subsection{Results}\label{temporal-results}

We show an IPC quality table and a quality plot of an experimental run of these planners on the tempo-sat-6 dataset (Figure~\ref{fig:tempo-sat-6-results}).
Additionally, sample plan Gantt charts are shown in Figure~\ref{fig:tempo-sat-6-gantt}. The generated plans and benchmark results can be found in the attached \nameref{cd-contents}.


\begin{figure}[p]
\centering
\begin{subtable}{\textwidth}
\centering
\scriptsize
\renewcommand{\footnotesize}{\scriptsize}
\begin{tabular}{lrrrrrrr}
\toprule
\textbf{\#} & \textbf{MSFA5Sched} & \textbf{RRAPNSched} & \textbf{TFD2014} & \textbf{TRRAPN} & \textbf{sgplan6} & \textbf{tfd} & \textbf{BEST}\\
\midrule
\multicolumn{1}{l|}{p01} & {\footnotesize 52} \textbf{1.00} & {\footnotesize 52} \textbf{1.00} & {\footnotesize 52.02} \textbf{1.00} & {\footnotesize 52} \textbf{1.00} & {\footnotesize 52} \textbf{1.00} & {\footnotesize 52} \textbf{1.00} & \multicolumn{1}{|r}{52}\\
\multicolumn{1}{l|}{p02} & {\footnotesize 125.01} \textbf{0.98} & {\footnotesize 126.01} \textbf{0.98} & {\footnotesize 150.11} \textbf{0.82} & {\footnotesize 126.01} \textbf{0.98} & {\footnotesize 217} \textbf{0.57} & {\footnotesize 208} \textbf{0.59} & \multicolumn{1}{|r}{123}\\
\multicolumn{1}{l|}{p03} & {\footnotesize 252.02} \textbf{0.75} & {\footnotesize 198.01} \textbf{0.95} & {\footnotesize 252.14} \textbf{0.75} & {\footnotesize 198.01} \textbf{0.95} & {\footnotesize 432} \textbf{0.44} & {\footnotesize 669} \textbf{0.28} & \multicolumn{1}{|r}{189}\\
\multicolumn{1}{l|}{p04} & {\footnotesize 341.02} \textbf{0.76} & {\footnotesize 260.02} \textbf{1.00} & {\footnotesize 425.29} \textbf{0.61} & {\footnotesize 267.01} \textbf{0.97} & {\footnotesize 845} \textbf{0.31} & uns. & \multicolumn{1}{|r}{260.02}\\
\multicolumn{1}{l|}{p05} & {\footnotesize 285.03} \textbf{0.85} & {\footnotesize 243.02} \textbf{1.00} & {\footnotesize 367.32} \textbf{0.66} & {\footnotesize 249.02} \textbf{0.98} & {\footnotesize 359} \textbf{0.68} & uns. & \multicolumn{1}{|r}{243.02}\\
\multicolumn{1}{l|}{p06} & {\footnotesize 316.02} \textbf{0.80} & {\footnotesize 253.01} \textbf{1.00} & {\footnotesize 408.31} \textbf{0.62} & {\footnotesize 265.02} \textbf{0.95} & {\footnotesize 965} \textbf{0.26} & uns. & \multicolumn{1}{|r}{253.01}\\
\multicolumn{1}{l|}{p07} & uns. & {\footnotesize 367.03} \textbf{1.00} & uns. & {\footnotesize 369.03} \textbf{0.99} & uns. & uns. & \multicolumn{1}{|r}{367.03}\\
\multicolumn{1}{l|}{p08} & uns. & {\footnotesize 481.04} \textbf{1.00} & uns. & {\footnotesize 532.04} \textbf{0.90} & uns. & uns. & \multicolumn{1}{|r}{481.04}\\
\multicolumn{1}{l|}{p09} & uns. & {\footnotesize 286.03} \textbf{1.00} & {\footnotesize 494.44} \textbf{0.58} & {\footnotesize 309.03} \textbf{0.93} & uns. & uns. & \multicolumn{1}{|r}{286.03}\\
\multicolumn{1}{l|}{p10} & uns. & uns. & {\footnotesize 939.8} \textbf{0.88} & {\footnotesize 827.07} \textbf{1.00} & uns. & uns. & \multicolumn{1}{|r}{827.07}\\
\multicolumn{1}{l|}{p11} & {\footnotesize 332.01} \textbf{1.00} & {\footnotesize 332.01} \textbf{1.00} & {\footnotesize 342.09} \textbf{0.97} & {\footnotesize 332.01} \textbf{1.00} & {\footnotesize 629} \textbf{0.53} & {\footnotesize 549} \textbf{0.60} & \multicolumn{1}{|r}{332}\\
\multicolumn{1}{l|}{p12} & {\footnotesize 483.01} \textbf{0.90} & {\footnotesize 490.01} \textbf{0.88} & {\footnotesize 543.13} \textbf{0.80} & {\footnotesize 490.01} \textbf{0.88} & {\footnotesize 817} \textbf{0.53} & {\footnotesize 982} \textbf{0.44} & \multicolumn{1}{|r}{433}\\
\multicolumn{1}{l|}{p13} & {\footnotesize 572.02} \textbf{0.68} & {\footnotesize 459.01} \textbf{0.85} & {\footnotesize 1172.38} \textbf{0.33} & {\footnotesize 434.01} \textbf{0.90} & {\footnotesize 650} \textbf{0.60} & {\footnotesize 3383} \textbf{0.11} & \multicolumn{1}{|r}{389}\\
\multicolumn{1}{l|}{p14} & {\footnotesize 777.03} \textbf{0.77} & {\footnotesize 621.02} \textbf{0.96} & {\footnotesize 1938.75} \textbf{0.31} & {\footnotesize 620.02} \textbf{0.96} & uns. & uns. & \multicolumn{1}{|r}{595}\\
\multicolumn{1}{l|}{p15} & {\footnotesize 1081.04} \textbf{0.76} & {\footnotesize 866.04} \textbf{0.95} & {\footnotesize 1143.45} \textbf{0.72} & {\footnotesize 860.04} \textbf{0.96} & {\footnotesize 2249} \textbf{0.37} & uns. & \multicolumn{1}{|r}{824}\\
\multicolumn{1}{l|}{p16} & {\footnotesize 1532.07} \textbf{0.49} & {\footnotesize 760.03} \textbf{0.98} & {\footnotesize 2198.97} \textbf{0.34} & {\footnotesize 752.03} \textbf{0.99} & {\footnotesize 1875} \textbf{0.40} & uns. & \multicolumn{1}{|r}{748}\\
\multicolumn{1}{l|}{p17} & {\footnotesize 1317.07} \textbf{0.60} & {\footnotesize 906.03} \textbf{0.87} & {\footnotesize 2393.97} \textbf{0.33} & {\footnotesize 916.04} \textbf{0.86} & {\footnotesize 3331} \textbf{0.24} & uns. & \multicolumn{1}{|r}{789}\\
\multicolumn{1}{l|}{p18} & {\footnotesize 1960.09} \textbf{0.62} & {\footnotesize 1217.05} \textbf{1.00} & uns. & {\footnotesize 1224.06} \textbf{0.99} & uns. & uns. & \multicolumn{1}{|r}{1217.05}\\
\multicolumn{1}{l|}{p19} & {\footnotesize 2226.12} \textbf{0.56} & {\footnotesize 1266.06} \textbf{0.99} & uns. & {\footnotesize 1254.06} \textbf{1.00} & uns. & uns. & \multicolumn{1}{|r}{1254.06}\\
\multicolumn{1}{l|}{p20} & {\footnotesize 2596.13} \textbf{0.42} & {\footnotesize 1399.07} \textbf{0.77} & uns. & {\footnotesize 1488.08} \textbf{0.73} & {\footnotesize 6362} \textbf{0.17} & uns. & \multicolumn{1}{|r}{1084}\\
\multicolumn{1}{l|}{p21} & {\footnotesize 94.02} \textbf{0.67} & {\footnotesize 69.01} \textbf{0.91} & {\footnotesize 102.14} \textbf{0.62} & {\footnotesize 69.01} \textbf{0.91} & {\footnotesize 113} \textbf{0.56} & {\footnotesize 161} \textbf{0.39} & \multicolumn{1}{|r}{63}\\
\multicolumn{1}{l|}{p22} & {\footnotesize 192.03} \textbf{0.49} & {\footnotesize 114.01} \textbf{0.82} & {\footnotesize 265.38} \textbf{0.35} & {\footnotesize 114.01} \textbf{0.82} & {\footnotesize 238} \textbf{0.39} & uns. & \multicolumn{1}{|r}{94}\\
\multicolumn{1}{l|}{p23} & {\footnotesize 278.04} \textbf{0.44} & {\footnotesize 156.02} \textbf{0.79} & {\footnotesize 342.44} \textbf{0.36} & {\footnotesize 156.02} \textbf{0.79} & {\footnotesize 642} \textbf{0.19} & uns. & \multicolumn{1}{|r}{123}\\
\multicolumn{1}{l|}{p24} & {\footnotesize 262.04} \textbf{0.53} & {\footnotesize 184.02} \textbf{0.76} & uns. & {\footnotesize 184.02} \textbf{0.76} & {\footnotesize 1116} \textbf{0.13} & uns. & \multicolumn{1}{|r}{140}\\
\multicolumn{1}{l|}{p25} & {\footnotesize 373.05} \textbf{0.42} & {\footnotesize 199.02} \textbf{0.78} & uns. & {\footnotesize 191.02} \textbf{0.82} & uns. & uns. & \multicolumn{1}{|r}{156}\\
\multicolumn{1}{l|}{p26} & uns. & {\footnotesize 234.03} \textbf{1.00} & uns. & {\footnotesize 234.02} \textbf{1.00} & uns. & uns. & \multicolumn{1}{|r}{234.02}\\
\multicolumn{1}{l|}{p27} & uns. & {\footnotesize 254.03} \textbf{1.00} & uns. & {\footnotesize 256.03} \textbf{0.99} & uns. & uns. & \multicolumn{1}{|r}{254.03}\\
\multicolumn{1}{l|}{p28} & uns. & {\footnotesize 312.03} \textbf{1.00} & uns. & {\footnotesize 314.03} \textbf{0.99} & uns. & uns. & \multicolumn{1}{|r}{312.03}\\
\multicolumn{1}{l|}{p29} & uns. & {\footnotesize 314.03} \textbf{1.00} & uns. & {\footnotesize 314.03} \textbf{1.00} & uns. & uns. & \multicolumn{1}{|r}{314.03}\\
\multicolumn{1}{l|}{p30} & uns. & {\footnotesize 385.04} \textbf{0.90} & uns. & {\footnotesize 346.03} \textbf{1.00} & uns. & uns. & \multicolumn{1}{|r}{346.03}\\
\midrule
\textbf{total} & \textbf{14.50} & \textbf{27.16} & \textbf{11.05} & \textbf{28.02} & \textbf{7.35} & \textbf{3.43} & \\
\bottomrule
\end{tabular}


\caption{Quality and score of sequential planners on the tempo-sat-6 dataset.}
\label{tab:tempo-sat-6-ipc-scores}
\end{subtable}

\vspace{0.5cm}
\begin{subfigure}{\textwidth}
\centering
\includegraphics[width=1.0\textwidth]{../imga/tempo-sat-6-quality}
\caption{Quality plot of sequential planners on the tempo-sat-6 dataset.}
\label{fig:tempo-sat-6-quality}
\end{subfigure}
\caption{Planner results on tempo-sat-6.}
\label{fig:tempo-sat-6-results}
\end{figure}

\begin{figure}
\centering
\begin{subfigure}{\textwidth}
\centering
\includegraphics[width=1.0\textwidth]{../imga/tempo-sat-6-gantt-p15-A}
\caption{Gantt chart of the temporal planner \TODO{A} on the tempo-sat-6 \texttt{p15} problem.}
\label{fig:tempo-sat-6-gantt-15-a}
\end{subfigure}

\begin{subfigure}{\textwidth}
\centering
\includegraphics[width=1.0\textwidth]{../imga/tempo-sat-6-gantt-p15-B}
\caption{Gantt chart of the temporal planner \TODO{B} on the tempo-sat-6 \texttt{p15} problem.}
\label{fig:tempo-sat-6-gantt-15-b}
\end{subfigure}
\caption{Gantt charts of the temporal planner \TODO{A and B} on the tempo-sat-6 \texttt{p15} problem.}
\label{fig:tempo-sat-6-gantt}
\end{figure}

\subsection{Discussion}

Planners that entered the 2008 temporal track at the IPC did not cope well with the Transport domain
--- only two non-baseline planners were able to produce at least one plan
for any problem. Additionally, only the simple problem (\verb+p01+) was solved
optimally by any planner. The best total quality was only $7.5/30$, achieved by
\textit{SGPlan$_6$}. No other domain in the temporal track had a lower best total quality
than Transport, which, assuming reasonably generated problem instances, hints
at Transport being one of the harder domains for domain-independent temporal planners.

\TODO{Vytiahnut si plany z vysledkov seq-sat08 TFD a sgplan6, najst deficiencies}

\TODO{discuss results}


\section{Overall results}

\TODO{intro + recap}