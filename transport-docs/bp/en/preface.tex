\chapter*{Introduction}
\addcontentsline{toc}{chapter}{Introduction}

With the advent of self-driving cars and automated warehouses transportation has become an even more essential topic than it was a few years ago.
Cheaper air, bus, and rail travel has enabled people to travel more than they have ever before.
Every day, thousands of companies need to deliver items from
one place to the other. Due to economic, ecological, and safety reasons, making the use of transportation resources more efficient is essential not just for saving money,
but for the future of our planet.
It is only natural to try to automate the planning processes for various transportation problems.
We want to calculate the shortest possible paths for our vehicles to take,
make the calculated plans adhere to laws or regulations, and possibly even accommodate the drivers'
wishes. Doing all of this by hand is a quite non-trivial task -- today's computers
are far more effective for solving these problems.

A major part of solving these problems, \textit{planning}, is usually defined as the reasoning side of acting \citep[Section~1.1]{Ghallab2004}.
This means that we want to come up with a \textit{sequence of actions} that leads to a desired \textit{goal state}.
In transportation, examples of planning problems include finding a sequence of orders for truck drivers to pick up and deliver packages in a given day,
or orchestrating robots in a warehouse to wrap and transport ordered goods to the loading dock for shipping.

Automated planning has historically been focused on \textit{domain-independent} planning -- planning without
the use of specific knowledge about the problem's domain.
As \citet{Nau2007} states, this is mostly due to the research field of planning wanting to
establish itself generally -- focusing on a set of domains would not be useful for that.
They also believe that this bias against domain-dependent planning is not as useful anymore.
We can now benefit from the attained theoretical results and along with an advancement in computing power,
a wider range of practical problems can now be solved using planning techniques.

In this thesis, we will study variants of a specific planning domain introduced in the 2008 International Planning Competition (IPC) called \textit{Transport}. It serves as an abstract representation of a family of related transportation problems
and an important benchmark for planning. Using several variants
of this domain, we will compare the performance of our custom-built planners to that of the planners taking part in the original competition and discuss various advantages or shortcomings of these approaches.

The Transport domain, in its basic form, consists of a road network with items located at specified locations. The items are to be delivered to their destinations
using a fleet of vehicles. Our aim is to deliver all items
with the least total cost, or in the shortest amount of time,
where the cost and/or duration of individual actions is dependent on the domain variant.

A natural interpretation of the Transport domain is that it represents a set of trucks
delivering packages. However,
the exact same domain formulation could be used to
model a ride-sharing service like Uber, where cars drive around a city, picking up and dropping off people along the way, or
a means of modeling a rush-hour scenario in a public transportation system.

To aid in the construction of planners and analysis of generated
plans, we will also develop a planning system called TransportEditor. It will consist of a problem visualizer and editor for the Transport domain. To give insight into plan deficiencies, it will be possible to trace plan actions and see how the planning
state evolves as the plan is executed. For fast prototyping,
it will be possible to generate plans by using built-in and external planners without leaving the system. Currently, only a handful of similar systems exist and to the best of our knowledge, none of them is specialized for problems modeling a graph with agents on it.

We aim to show that domain-dependent planning has an important
role to play in the future and there are many problems to be solved,
despite the loss of generality when compared to domain-independent planning.
To show this, we will design custom planners
and try to come up with domain-specific heuristics and other features
to aid our planners in solving the problems as well and as fast as
possible.

First, we will formalize and define the specific form of our
chosen planning domain and its variants. We will show how
it compares to other similarly themed problems that are being
solved. Afterward, we will describe the approaches we used
to build planners for both a
simple, sequential transport domain and a more complicated
temporal variant. Finally, we will design and run benchmarks,
comparing the performance of our planners to the current
state of the art domain-independent planners.
