\chapter*{Introduction}
\addcontentsline{toc}{chapter}{Introduction}

With the advent of self-driving cars, automated warehouses and \TODO{remote navigation systems},
transportation has become an even more current topic than it was a few years ago.
Cheaper bus, rail and air travel has enabled people to travel more than they have ever before.
Every day, thousands of companies need to deliver items from
one place to the other. For economic, ecological, safety and a myriad of other reasons, making the use of transportation resources more efficient is bread and butter for a lot of people. It is only natural to try to automate the accompanying planning processes and make them faster, cheaper and of higher quality.

Planning, as a process, is usually defined as the reasoning side of acting \citep[Section~1.1]{Ghallab2004}, meaning that we want to come up with some \textit{sequence of actions}, leading to a desired \textit{goal state}. In transportation, examples of planning problems include finding a sequence of orders for truck drivers to pick up and deliver packages for the day, or orchestrating robots in a warehouse to deliver ordered goods to the loading dock for shipping.

Automated planning has historically been more focused on
classical, \textit{domain-independent} planning -- planning without
the use of specific knowledge about the problem's domain.
This is mostly due to the field of planning wanting to
establish itself among other, larger fields \TODO{cite Nao: planning review}. As a result, deeply theoretical results
have been obtained and along with an advancement in computing power, can now be used to apply planning
to a wider range of practical problems.

In this thesis, we will study variants of a specific planning domain introduced in the 2008 International Planning Competition (IPC) called \textbf{Transport}. It serves as an abstract representation of a family of related transportation problems
and an important benchmark for planning. On several variants
of this domain, we will compare the results of our custom-built planners to the original competition results and discuss
various shortcomings and advantages of both approaches.

The Transport domain, in its basic form, consists of a road network with items located at specified locations. The items are to be delivered to their destinations
using a fleet of vehicles. Our aim is to deliver all items
with the least total cost, or in the shortest amount of time, depending on the specific domain variant.

This domain has naturally been interpreted as a set of trucks
delivering packages. However,
the exact same domain formulation could be used to
model a ride-sharing service, with cars driving around a city, picking up and dropping off people along the way, or
a means of modeling a public transportation system setup using
data from a morning rush hour.

To aid in the construction of planners and analysis of generated
plans, we will also develop a planning system called TransportEditor. It will consist of a problem visualizer and editor for the Transport domain. To give insight into plan deficiencies, it will be possible to trace plan actions and see how the planning
state evolves as the plan is executed. For fast prototyping,
it will be possible to generate plans by using built-in and external planners without leaving the system. Currently, there only a handful of similar systems exist and to our best knowledge, none of them is specialized for problems containing a graph with
agents on it.

We aim to show that domain-dependent planning has an important
role to play in the future and there are still problems to be solved, despite the loss of generality compared to domain-independent planning.
To show this, we will design custom planners
and try to come up with domain-specific heuristics and other tricks
to aid the planners in solving the problems as best and as fast as
it is possible.

First, we will formalize and define the specific form of our
chosen planning domain and its variants. We will show how
it compares to other similarly themed problems that are being
solved. Afterward, we will describe the approaches we used
to build planners for both a
simple, sequential transport domain and a more complicated
temporal variant. Finally, we will design and run benchmarks,
comparing the performance of our planners to the current
state of the art domain-independent planners.
