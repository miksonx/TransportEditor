\chapter{Temporal Transport planning}

The temporal domain not only has the added challenge of time,
but fuel demands and vehicle target locations are also present.
In this chapter, we describe approaches used
for tackling these additional challenges.















\section{Scheduling actions of sequential plans}\label{sched}

A simple temporal planning
technique that is surprisingly effective in practice
is one that simply forgets about time,
finds a plan, and reintroduces time and concurrency as an afterthought
into the generated plan.

To be able to precisely formulate our algorithm, we
have to define the concept of a mutex first.
We say that a pair of instantiated temporal operators $a$, $b$
is in a mutual exclusion relation (mutex)
if and only if $a$ and $b$
cannot overlap in a valid plan.
For example, a pair of \pickup{} actions 
of the same vehicle is mutex, because the
at start effect of each action is to 
set the \verb+ready-loading+
predicate to false and the action has an at start
condition that requires \verb+ready-loading+
to be true.

Leveraging our domain knowledge, we pre-construct mutex relations for Transport. The following actions are mutex:
\begin{itemize}
\item any pair of actions
of the same vehicle, except a \refuel{} and \pickup{}/\drop{} pair (in any order); and
\item a \drop{} and \pickup{} pair (in any order)
of the same package.
\end{itemize}

Our algorithm starts with relaxing the temporal
problem to a sequential one by removing fuel demands
and any notion of time.
After running a sequential planner on the relaxed problem,
we schedule the plan by finding a
topological ordering of a directed acyclic graph (DAG)
of actions, where edges are specified by mutex relations.
Using the graph, we add actions to the temporal plan
at the earliest available time, based on the topological
order.

The DAG described above is constructed
from a sequential plan $\pi$ by adding
all actions of the plan as nodes of the graph
and all mutex relations as edges.
The direction of added edges is based on the order
the actions of a mutex appear in the original plan $\pi$.

To find a topological ordering in the mutex DAG,
we use the algorithm described in \citet{Kahn1962}.
We traverse actions of the sequential plan $\pi$
(nodes of the DAG)
in topological order and add each action $a$
to the temporal plan $\pi^T$
at the maximum end time in $\pi^T$ of all actions
who are mutex with $a$, where $a$ is second in the pair.
Thanks to topological ordering, all of these actions
are already planned in $\pi^T$ and hence we know their end times.
The described algorithm is summarized in Figure~\ref{alg:sched}.

\myalg{Sequential plan scheduling}%
{%
\Input a temporal Transport problem $\mathcal{P} = (S, O, \gamma, s_0, g),$
a sequential plan $\pi$
\Output a temporal plan $\pi^T$
\comment{
\Function{Refuel-Positions}{$v$, $\pi$}
\State $P_v \gets $ empty list
\For{$i \gets 1 \textbf{ to } |\pi|$}
\State $a \gets \pi[i]$
\If{$a$ is a \drive{} action of vehicle $v$}
\If{$\mt{has-petrol-station}(from(a))$} \Comment{The drive's source location}
\State $P_v \gets$ append $i$ to $P_v$
\EndIf
\EndIf
\EndFor
\State \Return $P_v$
\EndFunction

\Function{Add-Refuels}{$k$, $P_v$, $\pi$}

\State \Return $\pi$
\EndFunction

\Function{Add-Refuel-Actions}{$\mathcal{P}$, $\pi$}
\ForAll{vehicle $v$}
\State $P_v \gets $ \Call{Refuel-Positions}{$v$, $\pi$}
\State $F \gets $ failure
\For{$k \gets 0 \textbf{ to } |\pi|$}
\State $F \gets $ \Call{Add-Refuels}{$k$, $P_v$, $\pi$}
\If{$F$ \textbf{is not} failure}
\State $\pi \gets F$
\State \textbf{break}
\EndIf
\EndFor
\If{$F$ \textbf{is} failure}
\Return failure
\EndIf
\EndFor
\State \Return $\pi$
\EndFunction
}
\Function{Build-Mutex-Graph}{$\pi$}
\State $G \gets $ empty directed graph
\ForAll {action $a \in \pi$} 
\State $G \gets $ add node $a$ to $G$
\EndFor
\For{$i \gets 1 \textbf{ to } |\pi|$}
\State $a \gets \pi[i]$
\For{$j \gets i \textbf{ to } |\pi|$}
\State $b \gets \pi[j]$
\If{$veh(a) = veh(b)$ and not \pickup{}/\drop{} and \refuel{}}
\State $G \gets $ add edge $(a, b)$ to $G$
\ElsIf{$a$ and $b$ are \pickup{} and \drop{} actions}
\State $G \gets $ add edge $(a, b)$ to $G$
\EndIf
\EndFor
\EndFor
\State \Return $G$
\EndFunction

\Function{Schedule}{$\mathcal{P}$, $\pi$, $\delta$ ($= 0.001$)}
\State $\pi^T \gets $ empty temporal plan, $t_s, t_e \gets $ empty maps
\State $\pi \gets $ \Call{Add-Refuel-Actions}{$\mathcal{P}$, $\pi$}
\If{$\pi$ \textbf{is} failure} \Return failure \EndIf
\State $G \gets $ \Call{Build-Mutex-Graph}{$\pi$}
\State $G' \gets $ \Call{Topological-Sort}{$G$}

\ForAll{action $b \in G'$}
\State $t_s[b] \gets \delta + \max_{(a, b) \in G'} t_e[a]$
\Comment{If no such edge exists, defaults to 0}
\State $t_e[b] \gets t_s[b] + duration(b)$
\State $\pi^T \gets $ append $(t_s[b], t_e[b], b)$ to $\pi^T$
\EndFor
\If{$\exists$ vehicle $v : \mt{fuel-left}(v) < 0$ at any point during $\pi^T$}
\Return failure
\EndIf
\State \Return $\pi^T$
\EndFunction
}%
{A scheduling algorithm
for temporal planning using sequential planners.}{A scheduling algorithm
for temporal planning using sequential planners.
The maps $t_s$ and $t_e$ represent the assigned start and end times (respectively) of planned actions. The parameter $\delta > 0$ is a constant used for separating different actions.
\textsc{Topological-Sort} returns the same graph, but with nodes sorted
using Kahn's algorithm.
\textsc{Add-Refuel-Actions} is briefly described in Section~\ref{sched}.}{sched}{tb}

Previously left out was an important step of the algorithm,
which consists of adding fuel constraints back into the plan.
Just before constructing the mutex DAG,
we attempt to add the least amount of \refuel{} actions
into the plan, so that no fuel capacity constraints are broken.
This is done iteratively for each vehicle,
by looking at combinations of possible places to refuel for that vehicle in a plan.
This approach, however, does not always lead to a valid plan,
because not all
sequential plans go through enough locations that have a petrol station
to allow all vehicles to be sufficiently fueled.
The problem is solved by simply giving up on the current plan
and attempting to generate another sequential plan.
We keep track of the best-known plan and always update it
if we find a better scheduled and valid plan.
Despite this, some combinations of sequential planners
and specific problem instances might never generate a sequential plan
that when scheduled produces a valid temporal plan (due to fuel capacities of vehicles).

A big advantage of using a scheduling approach to temporal planning
is that it is possible to
reuse sequential planners which have already been created
and optimized.
The choice of an underlying planner makes a big difference
in the quality of scheduled plans. Planners
that generate various different plans generally perform better
in such a scenario, because the scheduler has more
opportunities to find an extension of the plan with respect to fuel demands.

Also, suboptimal plans in the sequential domain
sometimes yield better scheduled plans than optimal ones.
For example, consider a triangular road network with three locations $A$, $B$, and $C$, connected by roads of length 1.
There are two packages $p_A$, $p_B$ and two vehicles $v_A$, $v_B$,
positioned at $A$ and $B$ respectively. The target location of both
packages is $C$. The capacities and fuel capacities of all vehicles
are 2 and both packages have a size of 1.
In the sequential variant, the optimal plan consists of either
vehicle picking up the package at their location, driving
over to the location with the other package, and delivering both packages
to $C$. The cost of such a plan is $6$.
However, in the temporal variant, both of the vehicles can drive at the same time, meaning that the optimal plan makespan is only $3$.
If the sequential planner never generates a plan
using both vehicles, the temporal scheduler has no way of
parallelizing the generated actions, hence it will never
schedule a plan with a lower makespan than $6$.

We will test the scheduling algorithm with a Randomized Restart planner (Section~\ref{rand-restart})
and a meta-heuristically weighted forward planner (Section~\ref{msfa}) in Chapter~\ref{experiments}.



























\section{TODO: Ad-hoc temporal planning?}\label{temporal-approach}
