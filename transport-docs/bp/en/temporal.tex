\chapter{Temporal Transport planning}

The temporal domain not only has the added challenge of time,
but fuel demands and vehicle target locations are also present.
In this chapter, we describe approaches used
for tackling these challenges.















\section{Scheduling actions of sequential plans}\label{sched}

A simple temporal planning
technique that is suprisingly effective in practice
is one that simply forgets about time,
finds a plan, and reintroduces time and concurrency as an afterthought
into the generated plan.

In the case of Transport, we simply relax the temporal
problem to a sequential one by removing fuel demands
and any notion of time.
After running a sequential planner on the relaxed problem,
we schedule the plan by finding a
topological ordering of the directed acyclic graph (DAG)
of mutexes. Using the graph, we add actions to the temporal plan
at the earliest available time, based on the topological
order and mutex relations.

We say that a pair of instantiated temporal operators $a$, $b$
is in a mutual exclusion relation (mutex)
if and only if $a$ and $b$
cannot overlap in a valid plan.
For example, a pair of \pickup{} actions 
of the same vehicle is mutex, because the
at start effect of each actions is to 
set the \verb+ready-loading+
predicate to false and the action has an at start
condition that requires \verb+ready-loading+
to be true.

Leveraging our domain knowledge, we pre-construct mutex relations for Transport. The following actions are mutex:
\begin{itemize}
\item any pair of actions
of the same vehicle, except a \refuel{} and \pickup{}/\drop{} pair (in any order); and
\item a \drop{} and \pickup{} pair (in any order)
of the same package.
\end{itemize}

The DAG described above is constructed
from a sequential plan $\pi$ by adding
all actions of the plan as nodes of the graph
and all mutex relations as edges.
The edges are directed based on the order the actions appear in the original plan $\pi$.

We find a topological ordering in the mutex DAG
using the algorithm described in \citet{Kahn1962}.





 and starting at time 0,
iterate through the actions
.



\TODO{choice of seq planner}

\TODO{describe the idea of parallelizing sequential plans using ``mutexes''}

\TODO{easy to implement, leveraging existing work + describe shortcommings}





























%\section{Ad-hoc temporal planning}\label{temporal-approach}
%\TODO{temporal RRAPN}
