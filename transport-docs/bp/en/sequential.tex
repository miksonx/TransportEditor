\chapter{Sequential Transport planners}

\TODO{Specifics of sequential planning}

\section{State-space forward planning}

\TODO{Forward planning}

\TODO{in state-space, Making it deterministic -> Search}

\TODO{describe the alg}

\TODO{describe a generic search alg}

\subsection{Simple distance-based heuristics}

\TODO{revise}
When designing a heuristic, we want to provide an as precise as possible estimate
of the total plan cost or duration.
In \texttt{transport-strips}, the only thing we want is to deliver packages to their destinations. Therefore, a straightforward heuristic is to calculate the lengths of a shortest
path of each package to its destination using the road network and sum them.
To make it more precise, we will add 1 if the package is currently loaded
in a vehicle or 2 if it is not loaded and not yet at its destination.
This heuristic is definitely not optimal, meaning that there are situations,
where we will need actions summing up to a higher cost.
However, it is important to note, that the previously mentioned state space heuristic 
is not even admissible.

\TODO{Improve this heuristic by applying the shortest vehicle path too + extend it for vehicle goals}

An \textit{admissible} heuristic does not overestimate
the true value it is approximating. During planning in state space,
when examining a state $s$, we want to estimate the total costs of the best
plan getting us to a goal state from state $s$. In other words, because we are
trying to minimize the total cost,
a planning heuristic $h: S -> \N_0$ is admissible if and only if $\forall s \in S : h(s) \leq h^*(s),$
where $h^*$ is the true total cost. A similar definition is applicable for minimizing duration in the temporal variant.

Nonadmissible heuristics do not have nice properties when used with search algorithms
--- for example, they do not guarantee that the first path to a goal state we find
will be the optimal plan. To see how our previously constructed heuristic fails to do
this, imagine the situation of having a network with just two locations $A$ and $B$.
Two packages and one vehicle are located at $A$ and both packages want to be
transported to $B$. The road between $A$ and $B$ is symmetric and has length
of a 100. It is trivial to see that the optimal plan consists of two \pickup{} actions,
followed by a \drive{} and two \drop{}s. This plan has a total cost of $2+100+2=104$,
but the heuristic would estimate that we need 

\TODO{substitute drive actions for higher level drives to any node on graph, but forbid two sequential high level drives.. adv: always uses shortest paths, never cycles, minimal overhead.. disadv: might throw off cooperation w.r.t the heuristic?}

\subsection{SFA*}

\TODO{describe BFS, describe how it is complete but too slow, etc} \citep[Section~3.5]{Russell1995}

\TODO{mention Zhou with Java impl} \citep{Zhou2015}

\TODO{Describe A* in forward planning}

\TODO{Advantages and shortcommings of SFA*}

\TODO{SFA* with a package distance heuristic}

\TODO{SFA* with a different heuristic?}

\TODO{WA*, W=5 napr}

\section{An ad-hoc planner based on something precalculated}

\section{Formulating Transport as a CSP}\label{csp-approach}

When discussing related works, we mentioned the Vehicle Routing Problem (Section~\ref{vrp}) and its straightforward
formulation as a Constraint Satisfaction Problem. Utilizing CSPs has been useful
for planning in the past \citep[Section~8.7]{Ghallab2004}
and we will attempt to use domain-specific knowledge to improve upon the standard, domain-independent
formulation.

\subsection{Na{\"{i}}ve CSP formulation}

We will now formulate a sequential Transport (Section~\ref{transport-strips}) problem as a CSP (Section~\ref{csp}) using the na{\"{i}}ve encoding provided in \citet[Section~8.3]{Ghallab2004}.


\TODO{explain the model + constraints}


However, using that strategy, our problems ``blow up'' in size --- as is expected due
to the different complexities of planning versus solving CSPs \citep[Section~8.3.2]{Ghallab2004}. To visualize the difference in our case, we have constructed a state space estimation table (Table~\ref{tab:csp-trivial}) for conversions of two sample sequential Transport problems.

\TODO{recalculate + verify numbers}

\begin{table}[tb]
\begin{center}
\begin{tabular}{l||rr}
\textbf{Features / estimates} & \textbf{p01} & \textbf{p20} \\ 
\midrule
\midrule
\textbf{Best known plan length} & 6 & 351 \\ 
\textbf{Vehicles} & 2 & 4 \\ 
\textbf{Vehicle variables} & 14 & 1 408 \\ 
\textbf{Packages} & 2 & 20 \\ 
\textbf{Package variables} & 14 & 7 040 \\ 
\textbf{Locations} & 5 & 60 \\ 
\textbf{Roads} & 12 & 256 \\
\textbf{Max capacity} & 4 & 4 \\ 
\midrule
\textbf{Ground Drive actions} & 168 & 360 448 \\ 
\textbf{Ground PickUp actions} & 140 & 1 689 600 \\ 
\textbf{Ground Drop actions} & 140 & 1 689 600 \\ 
\midrule
\textbf{Planning variables total} & 48 & 10 207 \\ 
\textbf{Grounded actions total} & 448 & 1 189 838 848 \\ 
\textbf{Search Space Estimate} & $\approx 1.1 \cdot 10^{52}$ & $\approx 1.4 \cdot 10^{27 952}$ \\ % https://www.wolframalpha.com/input/?i=(245120%5E351)+*+4%5E1408+*+60%5E1408+*+1468%5E7040
\end{tabular}
\end{center}
\caption[Search space approximations for a na{\"{i}}ve CSP encoding.]{CSP Search space approximations for the \textit{p01} and \textit{p20} problems from the \textit{seq-sat} track of IPC 2008, using the general and domain-independent encoding from \citet[Section~8.3]{Ghallab2004}.}
\label{tab:csp-trivial}
\end{table}

The first section of the table (rows 1--7) contains problem-specific constants.
The two calculated values in that section, \textit{Vehicle variables} and \textit{Package variables} are the numbers of variables generated for the respective
object by grounding it for every intermediate plan state (before and after applying an action). Therefore, the value is equal to the number of vehicles/packages of the problem
multiplied by the set plan length + 1 (each state corresponds to the state before applying an action + the last state).

In the second section (rows 8--10), we estimate the number of ground actions
Step 1 from \citet[Section~8.3.1]{Ghallab2004} will generate.
We calculate the number of \pickup{} and \drop{} actions the CSP encoding will generate
as $$(\mt{length(plan)} + 1) \cdot \mt{\#vehicles} \cdot \mt{\#locations} \cdot \mt{\#packages},$$
effectively counting all ground planning operators of the problem. Similarly,
the number of \drive{} actions is calculated as
$$(\mt{length(plan)} + 1) \cdot \mt{\#vehicles} \cdot \mt{\#roads},$$
which is more efficient than the na{\"{i}}ve way of
counting all
$$(\mt{length(plan)} + 1) \cdot \mt{\#vehicles} \cdot \mt{\#locations}^2$$
actions.

As we can see from the third section of the table, the number of variables
(planning variables and ground actions) is not extremely high
--- the problem is that the variables have very large domains,
which makes the CSP problem exponentially larger \citep[Section~8.3.2]{Ghallab2004}.
We calculated the \textit{Search Space Size Estimate} (SSE) as
\begin{align*}
\mt{SSE} =\; &\mt{\#ground\_actions}^{l-1} & \textit{\footnotesize select ground actions for the plan}\\
&\cdot \mt{\#capacities}^{l \cdot \mt{\#vehicles}} & \textit{\footnotesize select capacities for vehicle variables}\\
&\cdot \mt{\#locations}^{l \cdot \mt{\#vehicles}} & \textit{\footnotesize select locations for vehicle variables}\\
&\cdot (\mt{\#locations} + \mt{\#vehicles})^{l \cdot \mt{\#pkg}}, & \textit{\footnotesize select locations/vehicles for package variables}
\end{align*}
where we set $l := \mt{length(plan) + 1}$.
For comparison to the SSEs in the last table row, 
the estimated number of atoms in the universe is generally estimated to be about $4 \cdot 10^{80}$.

\subsection{Domain-dependent CSP representation}\label{csp-custom-repr}

We will now devise a different CSP representation for sequential Transport.
While not improving upon the search space estimates of the na{\"{i}}ve formulation
in a theoretical sense, we will describe a slightly more complex
and less general
CSP model that enables us to explore fewer states.

Using OptaPlanner and its shadow variable concept \citep[Section~4.3.6]{DeSmet2017}, we will model our Transport problem without keeping explicit track of capacities,
vehicle and package locations, and ground actions, all of which will be implicitly managed by OptaPlanner, or inferred in the case of capacity and actions.
This means we will reduce the memory overhead, maintaining the same expressive power
and hopefully not enlarge the processing time too much.

\TODO{Describe the specific representation after it is polished in the code}

\TODO{recalculate + verify numbers}

\begin{table}[tb]
\begin{center}
\begin{tabular}{l||rr}
\textbf{Features / estimates} & \textbf{p01} & \textbf{p20} \\ 
\midrule
\midrule
\textbf{Best known plan length} & 6 & 351 \\ 
\textbf{Vehicles} & 2 & 4 \\ 
\textbf{Vehicle shadow vars} & 14 & 1 408 \\
\textbf{Packages} & 2 & 20 \\ 
\textbf{Package shadow vars} & 14 & 7 040 \\
\textbf{Locations} & 5 & 60 \\
\textbf{Roads} & 12 & 256 \\
\textbf{Max capacity} & 4 & 4 \\ 
\midrule
\textbf{Ground Drive actions} & 24 & 1024 \\ 
\textbf{Ground PickUp actions} & 20 & 4800 \\ 
\textbf{Ground Drop actions} & 20 & 4800 \\ 
\midrule
\textbf{Planning variables total} & 48 & 10 207 \\ 
\textbf{Grounded actions in step} & 64 & 10 624 \\ 
\textbf{Action type orderings} & 2 187 & $\approx 8.8 \cdot 10^{167}$ \\ 
\textbf{Search Space Estimate} & $\approx 1.5 \cdot 10^{14}$ & $\approx 2.0 \cdot 10^{637}$ \\
\end{tabular}
\end{center}
\caption[Search space approximations for a domain-dependent CSP representation.]{CSP Search space approximations for the \textit{p01} and \textit{p20} problems from the \textit{seq-sat} track of IPC 2008, using a custom domain-dependent CSP representation for Transport sequential.}
\label{tab:csp-custom}
\end{table}

Using the domain-dependent representation specified, we are now able to construct
a search space estimate table for the same Transport problems (Table~\ref{tab:csp-custom}). 
Do note, that this table cannot be compared directly to the previous one,
because it hides the shadow variable management overhead.
Also, while the table rows look similar, sections 2 and 3 are calculated
differently. The ground action counts in section 2 are not multiplied by $\mt{length(plan)} + 1$
as done previously, because we only represent them once, not at every plan state.
The total number of grounded actions is the same, but they are not explicitly represented as variables. The Search Space size Estimate is therefore calculated differently:
\begin{align*}
\mt{SSE} =\; &3^{\mt{length(plan)} + 1} & \textit{\footnotesize select the action type of each action}\\
&\cdot \mt{\#ground\_actions}^{l-1}. & \textit{\footnotesize select the specific ground action}
\end{align*}
For comparison to the na{\"{i}}ve encoding SSEs which going from the p01 problem to p20 grow by a logarithmic factor of approximately $538$,
whereas the domain-dependent ones only grow by approximately $46$,
which is a huge improvement.

Given the search space reduction, we will now attempt to use this representation
for constructing a CSP-based planner.

\subsection{CSP-based planner}

\TODO{try to run a CSP in OptaPlanner to solve this and compare results, using Section~\ref{csp-custom-repr}}

\TODO{Advantages and shortcommings of a CSP-based planner}

\section{Summary}

\TODO{Mention which planners we choose to compare in the experimental part}

\TODO{more possible approaches}

