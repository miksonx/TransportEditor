%%%%%%%%%%%%%%%%%%%%%%%%%%%%%%%%%%%%%%%%%
% baposter Landscape Poster
% LaTeX Template
% Version 1.0 (11/06/13)
%
% baposter Class Created by:
% Brian Amberg (baposter@brian-amberg.de)
%
% This template has been downloaded from:
% http://www.LaTeXTemplates.com
%
% License:
% CC BY-NC-SA 3.0 (http://creativecommons.org/licenses/by-nc-sa/3.0/)
%
%%%%%%%%%%%%%%%%%%%%%%%%%%%%%%%%%%%%%%%%%

%----------------------------------------------------------------------------------------
%	PACKAGES AND OTHER DOCUMENT CONFIGURATIONS
%----------------------------------------------------------------------------------------

\documentclass[a0paper,fontscale=0.285]{baposter} % Adjust the font scale/size here

\usepackage{natbib}         % citation style AUTHOR (YEAR), or AUTHOR [NUMBER]
\setcitestyle{round} % round brackets for citep and citet

\usepackage{graphicx} % Required for including images
\graphicspath{{../img/}} % Directory in which figures are stored

\usepackage{amsmath} % For typesetting math
\usepackage{amssymb} % Adds new symbols to be used in math mode

\usepackage{booktabs} % Top and bottom rules for tables
\usepackage{enumitem} % Used to reduce itemize/enumerate spacing
\usepackage{palatino} % Use the Palatino font
\usepackage[font=small,labelfont=bf]{caption} % Required for specifying captions to tables and figures

\usepackage{multicol} % Required for multiple columns
\setlength{\columnsep}{1.5em} % Slightly increase the space between columns
\setlength{\columnseprule}{0mm} % No horizontal rule between columns

\setlist[itemize]{itemsep=1pt, topsep=0pt, parsep=0pt}
\setlist[description]{itemsep=1pt, topsep=0pt, parsep=0pt}
\setlist[enumerate]{itemsep=1pt, topsep=0pt, parsep=0pt}
%\newcommand{\compresslist}{ % Define a command to reduce spacing within itemize/enumerate environments, this is used right after \begin{itemize} or \begin{enumerate}
%\setlength{\itemsep}{1pt}
%\setlength{\parskip}{0pt}
%\setlength{\parsep}{0pt}
%}

\definecolor{lightblue}{rgb}{0.145,0.6666,1} % Defines the color used for content box headers

\title{\huge Planning for Transportation Problems} % Poster title

\author{Ondrej \v{S}kopek} % Author(s)

\newcommand{\insertdepartment}{Department of Theoretical Computer Science and Mathematical Logic}
\newcommand{\institute}{Faculty of Mathematics and Physics, Charles University} % Institution(s)

\begin{document}

\begin{poster}
{
headerborder=closed, % Adds a border around the header of content boxes
colspacing=1em, % Column spacing
bgColorOne=white, % Background color for the gradient on the left side of the poster
bgColorTwo=white, % Background color for the gradient on the right side of the poster
borderColor=white, % Border color
headerColorOne=lightblue, % Background color for the header in the content boxes (left side)
headerColorTwo=lightblue, % Background color for the header in the content boxes (right side)
headerFontColor=white, % Text color for the header text in the content boxes
boxColorOne=white, % Background color of the content boxes
textborder=rectangle, % Format of the border around content boxes, can be: none, bars, coils, triangles, rectangle, rounded, roundedsmall, roundedright or faded
eyecatcher=false, % Set to false for ignoring the left logo in the title and move the title left
headerheight=0.1\textheight, % Height of the header
headershape=rectangle, % Specify the rounded corner in the content box headers, can be: rectangle, small-rounded, roundedright, roundedleft or rounded
headerfont=\Large\bf\textsc, % Large, bold and sans serif font in the headers of content boxes
%textfont={\setlength{\parindent}{1.5em}}, % Uncomment for paragraph indentation
linewidth=2pt % Width of the border lines around content boxes
}
%----------------------------------------------------------------------------------------
%	TITLE SECTION 
%----------------------------------------------------------------------------------------
%
{\includegraphics[height=8em]{logo-en.pdf}} % First university/lab logo on the left
{\vspace{0.2em}\huge\bf\textsc{Planning for Transportation Problems}\vspace{0.2em}%
} % Poster title
{\textsc{Ondrej {\v{S}}kopek\\\large\insertdepartment{}\\\institute{}}} % Author names and institution
{\includegraphics[width=21em]{logo-en.pdf}} % Second university/lab logo on the right

%----------------------------------------------------------------------------------------
%	OBJECTIVES
%----------------------------------------------------------------------------------------

\headerbox{Introduction}{name=introduction,column=0,row=0,span=1}{

Automated planning has historically been focused on \textit{domain-independent} planning. Our goal is to design \textit{domain-specific} planners 
for variants of the \textit{Transport} domain, introduced in the 2008 International Planning Competition (IPC).
Transport, in its basic form, consists of a road network with items located at specified locations. The items are to be delivered to their destinations
using a fleet of vehicles. Our aim is to deliver all items
with the least total cost, where the cost of individual actions is dependent on the domain variant.
}


%----------------------------------------------------------------------------------------
%	MATERIALS AND METHODS
%----------------------------------------------------------------------------------------

\headerbox{Materials \& Methods}{name=method,column=1,span=2,row=0,aligned=introduction}{ % This block's bottom aligns with the bottom of the conclusion block

\begin{multicols}{2}
We compare the performance of our planners to that of the planners taking part in the original competitions and discuss the results.

Sequential planners:
\begin{description}
\item[MSFA3] Meta-heuristically weighted SFA planner with the package and vehicle distance heuristic
\item[MSFA5] Meta-heuristically weighted SFA planner with the general marking heuristic
\item[RRAPN] Randomized Restart Around Path Nearby planner
\end{description}

Temporal planners:
\begin{description}
\item[MSFA5Sched] Scheduled MSFA5 planner
\item[RRAPNSched] Scheduled RRAPN planner
\item[TFD2014] Temporal Fast Downward planner, version 0.4 (external planner)
\item[TRRAPN] Temporal RRAPN planner
\end{description}

\vspace{1em}
Planners are benchmarked with:
\begin{itemize}
\item 30 minutes of CPU time;
\item 4 GB of RAM;
\item limited to one core per planner; and
\item run on the clusters of MetaCentrum.
\end{itemize}
\end{multicols}
}

%----------------------------------------------------------------------------------------
%	RESULTS
%----------------------------------------------------------------------------------------

\headerbox{Results}{name=results,column=0,span=3,below=introduction}{

\begin{multicols}{3}
\textbf{IPC 2008 --- Sequential satisficing} (Figure~1)
\vspace{0.15cm}

LAMA (overall IPC 2008 winner)
was the best planner on the sequential Transport domain, 
with a total quality of 28.93/30 (all others had less than 20).
After adding our planners, the total quality of LAMA drops to 24.36/30.

Our best planner on the this dataset, RRAPN:
\begin{itemize}
\item achieves a total quality of \textbf{27.85}/30;
\item is able to calculate
solutions of larger problems fast (7--10, 25--27); and
\item fails to achieve optimal scores
on smaller problems (2, 12),
due to its explicit nature.
\end{itemize}
MSFA3 and MSFA5 are similar in their construction and results on this dataset. They:
\begin{itemize}
\item obtain better results than RRAPN on smaller problems
(2--3, 21--22); and
\item are able to generate very good results even on larger problems (14--20, 28--29).
\end{itemize}
Based on total quality, MSFA5 marginally comes out on top as the better one of the two MSFA planners on this dataset.
All three of our planners beat all planners from the original competition based on total quality.

\vspace{0.15cm}
\textbf{IPC 2008 --- Temporal satisficing} (Figure~3)
\vspace{0.15cm}

Temporal planners did not cope well with the Transport domain
--- the best total quality was only 7.5/30 by SGPlan$_6$.
Our results show a performance increase when comparing TFD (from 2008) and TFD2014.

The three presented approaches yield improvements even over recent domain-independent temporal planners, like TFD2014:
\begin{itemize}
\item MSFA5Sched achieves a total of 9.25/30;
\item RRAPNSched achieves 27.18/30; and
\item TRRAPN beats both with 28.04/30.
\end{itemize}
MSFA5 does not generate plans of enough variety,
and therefore produces worse results when scheduled
(scheduler can't add enough refuel actions to make the plan feasible).

TRRAPN achieves the best total quality,
beating RRAPNSched by about 0.8,
even though it is marginally worse on some problems (4--7). TRRAPN gets better scores on
larger problems (10, 19, or 30), where RRAPNSched can't plan well with fuel.


\vspace{0.15cm}
\textbf{IPC 2011 --- Sequential satisficing} (Figure~2)
\vspace{0.15cm}

Only 4 planners achieved a total quality of more than 15/20, roamer was the winner on Transport overall.
Interestingly, LAMA 2008 was able to produce better plans than its 2011 version in 12/20 problems.

Results of our planners:
\begin{itemize}
\item RRAPN achieves better scores than all planners from the competition on all individual problems, with a total quality of 18.77/20;
\item the performance of MSFA planners
is almost indistinguishable, even on individual problems --- in total, MSFA3 achieves 18.78 and MSFA5 achieves 18.75; and
\item the performance of MSFA planners is complementary to RRAPN on some problems (3--6, 10--12, and 13--15).
\end{itemize}

\vspace{0.15cm}
\textbf{IPC 2014 --- Sequential satisficing} (Figure~4)
\vspace{0.15cm}

The winner on Transport was the Mercury planner, achieving
a stunning 20/20 total quality. The runner-up, yahsp3-mt, achieved a score of only 10.74/20
and all other planners achieved sub 10/20 total quality.

After adding the results of our planners:
\begin{itemize}
\item RRAPN manages to outperform yahsp3-mt with 15.91/20, yet it fails
to match the results of Mercury.
\item Both MSFA planners outperform RRAPN with qualities of 18.44 (MSFA3)
and 18.49 (MSFA5).

The results of MSFA3 and MSFA5 on this dataset are almost identical.
\end{itemize}
None of our planners came reasonably close to beating Mercury (at 19.25/20).
However, they do marginally outperform it on problems 4--7, 9--10, 12, and 18--19.
\end{multicols}

\begin{multicols}{2}
\begin{center}
\includegraphics[width=\linewidth]{../../bp/imga/seq-sat-6-quality.pdf}
\captionof{figure}{Quality chart of seq-sat-6.}
\end{center}
\begin{center}
\includegraphics[width=\linewidth]{../../bp/imga/seq-sat-7-quality.pdf}
\captionof{figure}{Quality chart of seq-sat-7.}
\end{center}

\begin{center}
\includegraphics[width=\linewidth]{../../bp/imga/tempo-sat-6-quality.pdf}
\captionof{figure}{Quality chart of tempo-sat-6.}
\end{center}
\begin{center}
\includegraphics[width=\linewidth]{../../bp/imga/seq-sat-8-quality.pdf}
\captionof{figure}{Quality chart of seq-sat-8.}
\end{center}

\end{multicols}
}

%----------------------------------------------------------------------------------------
%	REFERENCES
%----------------------------------------------------------------------------------------

\headerbox{References}{name=references,column=0,above=bottom,span=2}{

\renewcommand{\section}[2]{\vskip 0.05em} % Get rid of the default "References" section title
\nocite{*} % Insert publications even if they are not cited in the poster
\small{ % Reduce the font size in this block
\bibliographystyle{plainnat}
\bibliography{../../bp/en/bibliography-poster.bib} % Use sample.bib as the bibliography file
}}

%----------------------------------------------------------------------------------------
%	FUTURE RESEARCH
%----------------------------------------------------------------------------------------

\headerbox{Future Research}{name=futureresearch,column=2,span=1,above=references,below=results}{ % This block is as tall as the references block

There remain more approaches to apply to
to domain-specific planning:
\begin{itemize}
\item \textit{Hierarchical Task Networks} use \textit{tasks} (sequences of operators)
to carry out some goal. This approach is one of the most used in practice today (ad-hoc planners we designed are conceptually similar).

\item \textit{Pointer Networks and Reinforcement Learning} use special architectures of neural networks to solve TSP instances in \citet{Bello2016}.

\item \textit{Learning a domain-specific heuristic function} using neural networks \citep{Chen2011}. This
approach aims to help solve the problem of coming up with a good heuristic for a domain (a similar approach is also used in DeepStack \citep{Moravcik2017}).
\end{itemize}

}

%----------------------------------------------------------------------------------------
%	CONCLUSION
%----------------------------------------------------------------------------------------

\headerbox{Conclusion}{name=conclusion,column=0,span=2,above=references,below=results}{
The attained results show that domain-specific information can be leveraged
to generate plans of better quality.
We have designed and implemented Transport planners that are able to beat
all results from the sequential
and temporal satisficing tracks of the 2008 and 2011 IPCs.
In the 2014 IPC, we would have attained second place on overall quality in the Transport domain, behind the impressive result of Mercury. 
}

%----------------------------------------------------------------------------------------
%	Acknowledgements
%----------------------------------------------------------------------------------------

\headerbox{Acknowledgements}{name=ack,column=2,span=1,above=bottom,below=futureresearch}{
Thank you to prof.\;RNDr.\;Roman\;Bart{\'{a}}k, Ph.D., my advisor.

Provided access to computing facilities of MetaCentrum is greatly appreciated.
}



%----------------------------------------------------------------------------------------

\end{poster}

\end{document}
